\documentclass[12pt,a4paper]{article}
\usepackage[utf8]{inputenc}
\usepackage[T1]{fontenc}
\usepackage{kotex}            % 한글 지원
\usepackage{amsmath,amssymb}  % 수학 기호 및 수식 패키지
\usepackage{amsthm}           % 정리, 증명 환경
\usepackage{graphicx}         % 이미지 삽입
\usepackage{geometry}         % 페이지 여백 설정
\usepackage{hyperref}         % 하이퍼링크
\usepackage{color}            % 색상 지원

% 페이지 여백 설정
\geometry{margin=2.5cm}

% 정리, 정의, 예제 환경 설정
\newtheorem{theorem}{정리}[section]
\newtheorem{definition}{정의}[section]
\newtheorem{example}{예제}[section]
\newtheorem{lemma}{보조정리}[section]

% 증명 환경 한글화
\renewcommand{\proofname}{증명}

\title{LaTeX 수학 작성 가이드}
\author{Taeyang Lee}
\date{\today}

\begin{document}

\maketitle
\tableofcontents  % 목차 자동 생성
\newpage

% =====================================
\section{서론}

이 문서는 LaTeX를 사용하여 수학 문서를 작성하는 방법을 학습하기 위한 가이드입니다.
LaTeX는 수식을 아름답게 조판할 수 있는 강력한 도구로, 특히 수학과 과학 분야에서 널리 사용됩니다.
이 문서를 통해 기본적인 수식 작성부터 고급 기능까지 단계적으로 학습할 수 있습니다.

% =====================================
\section{기본 수식 작성법}

\subsection{인라인 수식과 디스플레이 수식}

LaTeX에서 수식을 작성하는 방법은 크게 두 가지로 나뉩니다.

\subsubsection{인라인 수식}
텍스트 중간에 수식을 넣을 때는 달러 기호를 사용합니다: $a^2 + b^2 = c^2$.
이렇게 작성하면 수식이 텍스트와 같은 줄에 나타납니다.
예를 들어, 원의 넓이는 $A = \pi r^2$로 표현할 수 있습니다.

\subsubsection{디스플레이 수식}
별도의 줄에 수식을 표시하고 싶을 때는 다음과 같이 작성합니다:
\[
    E = mc^2
\]
또는 번호가 있는 수식을 원한다면:
\begin{equation}
    \int_{0}^{\infty} e^{-x^2} dx = \frac{\sqrt{\pi}}{2}
\end{equation}


% =====================================
\section{자주 사용하는 수학 기호}

\subsection{기본 연산}
수학 문서 작성시 가장 자주 사용하는 기본 연산 기호들입니다:

\begin{itemize}
    \item 분수: $\frac{a}{b}$는 \verb|\frac{a}{b}|로 작성
    \item 제곱근: $\sqrt{x}$는 \verb|\sqrt{x}|로, $\sqrt[n]{x}$는 \verb|\sqrt[n]{x}|로 작성
    \item 위첨자/아래첨자: $x^2$는 \verb|x^2|로, $a_n$은 \verb|a_n|으로 작성
    \item 동시 사용: $x_i^2$는 \verb|x_i^2|로 작성
    \item 합: $\sum_{i=1}^{n} i = \frac{n(n+1)}{2}$
    \item 곱: $\prod_{i=1}^{n} i = n!$
\end{itemize}

\subsection{미적분}
미적분 관련 기호들은 다음과 같이 작성합니다:

\begin{itemize}
    \item 극한: $\lim_{x \to \infty} \frac{1}{x} = 0$
    \item 일반 미분: $\frac{df}{dx}$ 또는 $f'(x)$
    \item 편미분: $\frac{\partial f}{\partial x}$
    \item 정적분: $\int_{a}^{b} f(x) dx$
    \item 부정적분: $\int f(x) dx = F(x) + C$
\end{itemize}

\subsection{선형대수}
벡터와 행렬 표현법:

\begin{itemize}
    \item 벡터 표기법:
    \begin{itemize}
        \item 화살표 벡터: $\vec{v} = (v_1, v_2, v_3)$
        \item 굵은 글씨: $\mathbf{v} = (v_1, v_2, v_3)$
    \end{itemize}
    \item 2×2 행렬: $A = \begin{pmatrix} a & b \\ c & d \end{pmatrix}$
    \item 행렬식: $\det(A) = |A| = ad - bc$
    \item 전치행렬: $A^T$, 역행렬: $A^{-1}$
\end{itemize}

% =====================================
\section{정리와 증명 환경}

LaTeX는 수학 문서의 구조화를 위한 다양한 환경을 제공합니다.

\begin{definition}[소수]
    1보다 큰 자연수 중에서 1과 자기 자신만을 약수로 가지는 수를 \textbf{소수}(prime number)라고 한다.
\end{definition}

\begin{theorem}[피타고라스 정리]\label{thm:pythagoras}
    직각삼각형에서 빗변의 제곱은 나머지 두 변의 제곱의 합과 같다.
    \[
        c^2 = a^2 + b^2
    \]
\end{theorem}

\begin{proof}
    직각삼각형의 세 변을 $a$, $b$, $c$라 하자. 여기서 $c$는 빗변이다.
    넓이를 이용한 증명:
    \begin{align}
        (a+b)^2 &= a^2 + 2ab + b^2 \nonumber \\
        c^2 + 4 \cdot \frac{1}{2}ab &= (a+b)^2 \nonumber \\
        c^2 + 2ab &= a^2 + 2ab + b^2 \nonumber \\
        c^2 &= a^2 + b^2
    \end{align}
\end{proof}

\begin{lemma}
    모든 짝수의 제곱은 4의 배수이다.
\end{lemma}

\begin{example}
    이차방정식 $x^2 - 5x + 6 = 0$의 해를 구하시오.

    \textbf{풀이:}
    좌변을 인수분해하면:
    \begin{align*}
        x^2 - 5x + 6 &= (x-2)(x-3) = 0
    \end{align*}
    따라서 $x = 2$ 또는 $x = 3$이다.
\end{example}

% =====================================
\section{고급 수식 작성}

\subsection{여러 줄 수식 정렬}
align 환경을 사용하면 여러 줄의 수식을 깔끔하게 정렬할 수 있습니다:

\begin{align}
    (x+y)^3 &= (x+y)(x+y)^2 \\
            &= (x+y)(x^2 + 2xy + y^2) \\
            &= x^3 + 2x^2y + xy^2 + x^2y + 2xy^2 + y^3 \\
            &= x^3 + 3x^2y + 3xy^2 + y^3
\end{align}

\subsection{조건부 수식}
경우에 따라 달라지는 함수를 표현할 때:

\[
f(x) = \begin{cases}
    x^2 & \text{if } x \geq 0 \\
    -x^2 & \text{if } x < 0
\end{cases}
\]

\subsection{행렬 연산}
행렬의 곱셈 예제:

\[
\begin{pmatrix}
    1 & 2 \\
    3 & 4
\end{pmatrix}
\begin{pmatrix}
    a \\
    b
\end{pmatrix}
=
\begin{pmatrix}
    a + 2b \\
    3a + 4b
\end{pmatrix}
\]

% =====================================
\section{연습 문제}

\subsection{문제 1: 적분}
다음 적분을 계산하시오:
\[
    \int_0^1 x^2 dx
\]

\textbf{풀이:}
\begin{align}
    \int_0^1 x^2 dx &= \left[ \frac{x^3}{3} \right]_0^1 \tag{적분 공식} \\
                     &= \frac{1^3}{3} - \frac{0^3}{3} \nonumber \\
                     &= \frac{1}{3}
\end{align}

\subsection{문제 2: 급수}
다음 급수의 합을 구하시오:
\[
    \sum_{k=1}^{n} k = ?
\]

\textbf{풀이:}
가우스의 방법을 사용하면:
\begin{align*}
    S &= 1 + 2 + 3 + \cdots + n \\
    S &= n + (n-1) + \cdots + 1 \\
    2S &= (n+1) + (n+1) + \cdots + (n+1) \\
    2S &= n(n+1) \\
    S &= \frac{n(n+1)}{2}
\end{align*}

% =====================================
\section{유용한 팁}

\subsection{수식 작성 팁}
\begin{itemize}
    \item 여러 줄 수식은 \texttt{align} 환경 사용
    \item 수식 번호 제거는 \texttt{*} 추가 (예: \texttt{align*})
    \item 특정 줄만 번호 제거: \texttt{\textbackslash nonumber}
    \item 수동 번호 지정: \texttt{\textbackslash tag\{번호\}}
    \item 괄호 크기 자동 조절: \texttt{\textbackslash left(}, \texttt{\textbackslash right)}
    \item 수식 내 텍스트: \texttt{\textbackslash text\{...\}}
    \item 수식 참조: \texttt{\textbackslash label\{eq:name\}}, \texttt{\textbackslash eqref\{eq:name\}}
    \item 긴 분수: \texttt{\textbackslash dfrac} 대신 \texttt{\textbackslash frac} 사용
    \item 행렬 간격 조정: \texttt{\textbackslash arraystretch\{1.5\}}
\end{itemize}

\subsection{문서 구조 팁}
\begin{itemize}
    \item 섹션 자동 번호: \texttt{\textbackslash section}, \texttt{\textbackslash subsection}
    \item 번호 없는 섹션: \texttt{\textbackslash section*}
    \item 목차에만 추가: \texttt{\textbackslash addcontentsline\{toc\}\{section\}\{제목\}}
    \item 페이지 나누기: \texttt{\textbackslash newpage}, \texttt{\textbackslash pagebreak}
    \item 줄 간격 조정: \texttt{\textbackslash linespread\{1.5\}}
    \item 들여쓰기 설정: \texttt{\textbackslash setlength\{\textbackslash parindent\}\{20pt\}}
\end{itemize}

\subsection{한글 작성 팁}
\begin{itemize}
    \item 한글 폰트: \texttt{kotex} 패키지 사용
    \item 컴파일러: \texttt{xelatex} 또는 \texttt{lualatex} 사용
    \item 한글 문단 첫 줄 들여쓰기: \texttt{indentfirst} 패키지
    \item 한글 정리/정의 환경: \texttt{\textbackslash newtheorem\{theorem\}\{정리\}}
    \item 증명 환경 한글화: \texttt{\textbackslash renewcommand\{\textbackslash proofname\}\{증명\}}
\end{itemize}

\subsection{표와 그림 팁}
\begin{itemize}
    \item 표 생성: \texttt{tabular} 환경 사용
    \item 표 캡션: \texttt{\textbackslash caption} 명령어
    \item 표 위치 고정: \texttt{[H]} 옵션 (float 패키지 필요)
    \item 그림 삽입: \texttt{\textbackslash includegraphics[width=0.5\textbackslash textwidth]}
    \item 그림 나란히: \texttt{subfigure} 또는 \texttt{subcaption} 패키지
    \item 표/그림 참조: \texttt{\textbackslash ref\{fig:name\}}, \texttt{\textbackslash ref\{tab:name\}}
\end{itemize}

\subsection{참조와 링크 팁}
\begin{itemize}
    \item 스마트 참조: \texttt{\textbackslash usepackage\{cleveref\}}, \texttt{\textbackslash cref\{eq:name\}}
    \item 여러 참조: \texttt{\textbackslash cref\{eq:1,eq:2,eq:3\}}
    \item 페이지 참조: \texttt{\textbackslash pageref\{label\}}
    \item 외부 링크: \texttt{\textbackslash href\{URL\}\{텍스트\}}
    \item 이메일 링크: \texttt{\textbackslash mailto\{email@example.com\}}
    \item PDF 북마크: \texttt{hyperref} 패키지의 \texttt{bookmarks} 옵션
\end{itemize}

\subsection{컴파일 및 오류 해결 팁}
\begin{itemize}
    \item 컴파일 순서: LaTeX → BibTeX → LaTeX → LaTeX (참조 있을 때)
    \item 임시 파일 정리: \texttt{*.aux}, \texttt{*.log}, \texttt{*.toc} 삭제
    \item 오류 위치 찾기: 로그 파일에서 \texttt{!} 표시 확인
    \item 패키지 충돌: 패키지 로드 순서 변경 시도
    \item 인코딩 문제: UTF-8 인코딩으로 저장
    \item 긴 컴파일: \texttt{draft} 옵션으로 빠른 미리보기
\end{itemize}

\subsection{고급 기능 팁}
\begin{itemize}
    \item 매크로 정의: \texttt{\textbackslash newcommand\{\textbackslash mycmd\}[인수개수]\{정의\}}
    \item 조건부 컴파일: \texttt{ifthen} 패키지
    \item 외부 파일 포함: \texttt{\textbackslash input\{파일명\}}
    \item 하위 문서: \texttt{\textbackslash include\{파일명\}}
    \item 색상 사용: \texttt{\textbackslash textcolor\{색상\}\{텍스트\}}
    \item 박스 만들기: \texttt{\textbackslash fbox}, \texttt{\textbackslash colorbox}
    \item TikZ 그림: 복잡한 도형과 그래프 그리기
\end{itemize}

% =====================================
\section{결론}

이 문서를 통해 LaTeX의 기본적인 수식 작성 방법부터 정리 환경, 고급 기능까지 학습했습니다.
주요 학습 내용을 요약하면:

\begin{enumerate}
    \item 인라인 수식과 디스플레이 수식의 차이점
    \item 기본 수학 기호와 연산자 사용법
    \item 정리, 정의, 증명 환경의 활용
    \item 여러 줄 수식 정렬과 번호 제어
    \item 행렬과 조건부 수식 작성법
\end{enumerate}

LaTeX는 처음에는 복잡해 보일 수 있지만, 한 번 익숙해지면 매우 강력한 도구입니다.
이 가이드가 LaTeX 학습에 도움이 되기를 바랍니다.

\vspace{1cm}
\begin{center}
    \textit{Practice makes perfect!}
\end{center}

\end{document}