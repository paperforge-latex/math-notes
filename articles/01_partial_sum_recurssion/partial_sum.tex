\documentclass[12pt,a4paper]{article}
\usepackage[utf8]{inputenc}
\usepackage[T1]{fontenc}
\usepackage{kotex}            % 한글 지원
\usepackage{amsmath,amssymb}  % 수학 기호 및 수식 패키지
\usepackage{amsthm}           % 정리, 증명 환경
\usepackage{graphicx}         % 이미지 삽입
\usepackage{geometry}         % 페이지 여백 설정
\usepackage{hyperref}         % 하이퍼링크
\usepackage{color}            % 색상 지원

% 페이지 여백 설정
\geometry{margin=2.5cm}

% 정리, 정의, 예제 환경 설정
\newtheorem{theorem}{정리}[section]
\newtheorem{definition}{정의}[section]
\newtheorem{example}{예제}[section]
\newtheorem{lemma}{보조정리}[section]

% 증명 환경 한글화
\renewcommand{\proofname}{증명}

\title{부분합을 포함하는 점화식}
\author{Taeyang Lee}
\date{\today}

\begin{document}

\maketitle
\tableofcontents  % 목차 자동 생성
\newpage

% =====================================

\section{KMO 초급 대수 267번 문항}

\subsection{267번}
자연수 $n$에 대하여 모든 항이 양수인 수열 ${a_n}$이 $

\[
    \int_0^1 x^2 dx
\]

\subsection{풀이}
\begin{align}
    \int_0^1 x^2 dx &= \left[ \frac{x^3}{3} \right]_0^1 \\
                     &= \frac{1}{3} - 0 \\
                     &= \frac{1}{3}
\end{align}




% \section{기본 수식 작성법}

% \subsection{인라인 수식}

% 텍스트 중간에 수식을 넣을 때는 달러 기호를 사용합니다: $a^2 + b^2 = c^2$
% \begin{equation}
%     \int_{0}^{\infty} e^{-x^2} dx = \frac{\sqrt{\pi}}{2}
% \end{equation}

% \subsection{블록 수식}
% 별도의 줄에 수식을 표시할 때:
% \[
%     E = mc^2
% \]


% % =====================================
% \section{자주 사용하는 수학 기호}

% \subsection{기본 연산}
% \begin{itemize}
%     \item 분수: $\frac{a}{b}$
%     \item 제곱근: $\sqrt{x}$, $\sqrt[n]{x}$
%     \item 위첨자/아래첨자: $x^2$, $a_n$, $x_i^2$
%     \item 합/곱: $\sum_{i=1}^{n} i$, $\prod_{i=1}^{n} i$
% \end{itemize}

% \subsection{미적분}
% \begin{itemize}
%     \item 극한: $\lim_{x \to \infty} f(x)$
%     \item 미분: $\frac{df}{dx}$, $\frac{\partial f}{\partial x}$
%     \item 적분: $\int_{a}^{b} f(x) dx$
% \end{itemize}

% \subsection{선형대수}
% \begin{itemize}
%     \item 벡터: $\vec{v}$, $\mathbf{v}$
%     \item 행렬:
%     $\begin{pmatrix}
%         a & b \\
%         c & d
%     \end{pmatrix}$
%     \item 행렬식: $\det(A)$, $|A|$
% \end{itemize}

% % =====================================
% \section{정리와 증명 예제}

% \begin{definition}[소수]
%     1보다 큰 자연수 중에서 1과 자기 자신만을 약수로 가지는 수를 \textbf{소수}(prime number)라고 한다.
% \end{definition}

% \begin{theorem}[피타고라스 정리]
%     직각삼각형에서 빗변의 제곱은 나머지 두 변의 제곱의 합과 같다.
%     \[
%         c^2 = a^2 + b^2
%     \]
% \end{theorem}

% \begin{proof}
%     여기에 증명을 작성합니다.
%     \[
%         (a+b)^2 = a^2 + 2ab + b^2
%     \]
%     따라서 증명이 완료되었습니다.
% \end{proof}

% \begin{example}
%     $x^2 - 5x + 6 = 0$의 해를 구하시오.

%     \textbf{풀이:}
%     인수분해하면 $(x-2)(x-3) = 0$
%     따라서 $x = 2$ 또는 $x = 3$
% \end{example}

% % =====================================
% \section{연습 문제}

% \subsection{문제 1}
% 다음 적분을 계산하시오:
% \[
%     \int_0^1 x^2 dx
% \]

% \subsection{풀이}
% \begin{align}
%     \int_0^1 x^2 dx &= \left[ \frac{x^3}{3} \right]_0^1 \\
%                      &= \frac{1}{3} - 0 \\
%                      &= \frac{1}{3}
% \end{align}

% % =====================================
% \section{유용한 팁}

% \begin{itemize}
%     \item 여러 줄 수식은 \texttt{align} 환경 사용
%     \item 수식 번호 제거는 \texttt{*} 추가 (예: \texttt{align*})
%     \item 괄호 크기 자동 조절: \texttt{\textbackslash left(}, \texttt{\textbackslash right)}
%     \item 수식 내 텍스트: \texttt{\textbackslash text\{...\}}
% \end{itemize}

% =====================================
% 새로운 장을 추가하려면:
% \section{제목}
% 내용...

\end{document}