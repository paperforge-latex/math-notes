\documentclass[12pt,a4paper]{article}
\usepackage{kotex}            % 한글 지원
\usepackage{amsmath,amssymb}  % 수학 기호 및 수식 패키지
\usepackage{amsthm}           % 정리, 증명 환경
\usepackage{graphicx}         % 이미지 삽입
\usepackage{geometry}         % 페이지 여백 설정
\usepackage{hyperref}         % 하이퍼링크
\usepackage{color}            % 색상 지원
\usepackage{etoolbox}         % 환경 후킹을 위한 패키지
\usepackage{tikz}
\usetikzlibrary{fit,calc}

% 페이지 여백 설정
\geometry{margin=2.5cm}

% 정리, 정의, 예제 환경 설정
% \theoremstyle{definition}

\newtheoremstyle{test_form}% name
  {10pt}% Space above
  {10pt}% Space below
  {\normalfont\setlength{\parindent}{20pt}} % Body font(+ 본문 들여쓰기)
  {0em}% Indent amount
  {\bfseries}% Theorem head font
  {}% Punctuation after theorem head
  {\newline}% Space after theorem head (line break!)
  {}% Theorem head spec
\theoremstyle{test_form}
\newtheorem{problem}{문제}[section]
\newtheorem*{solution}{풀이}
\newtheorem*{answer}{정답}

% 섹션마다 문제 번호 리셋
\makeatletter
\@addtoreset{problem}{section}
\makeatother

% 문제 번호를 섹션 번호 없이 표시
\renewcommand{\theproblem}{\arabic{problem}}

% --- 자동 목차 등록 ---
\newcommand{\tocaddsolution}{%
  \phantomsection
  \addcontentsline{toc}{subsubsection}{풀이}%
}
\newcommand{\tocaddanswer}{%
  \phantomsection
  \addcontentsline{toc}{subsubsection}{정답}%
}

% problem 환경의 헤더 부분에 목차 추가
\makeatletter
\let\old@problem\problem
\renewcommand{\problem}{%
  \old@problem
  \phantomsection
  \addcontentsline{toc}{subsection}{문제 \theproblem}%
}
\makeatother

\AtBeginEnvironment{solution}{\tocaddsolution}
\AtBeginEnvironment{answer}{\tocaddanswer}

\title{KMO 대비반 중급 4주차 테스트}
\author{Taeyang Lee}
\date{\today}

\begin{document}

\maketitle
\tableofcontents  % 목차 자동 생성

\newpage

% =====================================
\section{대수}

\begin{problem}

자연수 \(a,b\)에 대해 연산 \(\star\)을 다음과 같이 정의하자. 
이때, \(3 \star (4 \star (5 \star \cdots \star (99 \star 100)))\cdots)\)은 
\(m!+n\)일 때, \(m+n\)을 구하여라.

\[
a \star b = ab - 2a - 2b + 6
\]

\begin{flushright}(20점)\end{flushright}

\end{problem}

\begin{solution}
\setlength{\parindent}{0pt}

주어진 연산을 다시 쓰면
\[
a \star b = ab - 2a - 2b + 6 = (a-2)(b-2) + 2.
\]

\(f(n) = n - 2\)로 치환하면, \(a = f(n) + 2\), \(b = f(m) + 2\)이므로
\[
a \star b = f(a) \cdot f(b) + 2.
\]

따라서
\[
\begin{aligned}
3 \star (4 \star (5 \star \cdots \star (99 \star 100)))
&= f(3) \cdot f(4 \star (5 \star \cdots)) + 2\\
&= 1 \cdot f(4 \star (5 \star \cdots)) + 2\\
&= f(4 \star (5 \star \cdots)) + 2.
\end{aligned}
\]

일반적으로, \(n \star (n+1 \star \cdots \star 100)\)을 계산하면
\[
\begin{aligned}
99 \star 100 &= (99-2)(100-2) + 2 = 97 \cdot 98 + 2\\
98 \star (99 \star 100) &= (98-2)(97 \cdot 98 + 2 - 2) + 2\\
&= 96 \cdot (97 \cdot 98) + 2\\
&= 96 \cdot 97 \cdot 98 + 2
\end{aligned}
\]

패턴을 확인하면
\[
n \star (n+1 \star \cdots \star 100) = (n-2)(n-1) \cdots 98 + 2.
\]

따라서
\[
3 \star (4 \star \cdots \star 100) = 1 \cdot 2 \cdot 3 \cdots 98 + 2 = 98! + 2.
\]

즉 \(m = 98\), \(n = 2\)이므로
\[
m + n = 98 + 2 = 100.
\]

\end{solution}

\begin{answer}
\hfill \boxed{100}
\end{answer}

\newpage

\begin{problem}

4차 다항식 \(p(x)\)가 \(p(k) = \dfrac{1}{k^2}\), \(k = 1, 2, \cdots, 5\)를 만족할 때, \(p(6)\)의 값을 기약분수로 표현했을 때, \(-\dfrac{n}{m}\)이라 하자. (\(m, n\)은 서로소인 정수) \(m+n\)을 구하여라.

\begin{flushright}(20점)\end{flushright}

\end{problem}

\begin{solution}

\setlength{\parindent}{0pt}

\textbf{접근: 보조 다항식과 비에타의 정리}

\(g(x) = x^2 p(x) - 1\)로 정의하면,
\[
g(k) = k^2 p(k) - 1 = k^2 \cdot \frac{1}{k^2} - 1 = 0 \quad (k = 1, 2, 3, 4, 5)
\]

따라서 \(g(x)\)는 \(x = 1, 2, 3, 4, 5\)를 근으로 가진다.

\(p(x)\)가 4차 다항식이므로 \(g(x)\)는 6차 다항식이다.

따라서 \(g(x)\)는 6개의 근을 가지며, 5개는 알려져 있고 나머지 하나를 \(\alpha\)라 하자.
\[
g(x) = A(x-1)(x-2)(x-3)(x-4)(x-5)(x-\alpha)
\]

\textbf{핵심 관찰: 1차항이 없음}

\(p(x) = c_4 x^4 + c_3 x^3 + c_2 x^2 + c_1 x + c_0\)로 놓으면,
\[
\begin{aligned}
g(x) = x^2 p(x) - 1 &= x^2(c_4 x^4 + c_3 x^3 + c_2 x^2 + c_1 x + c_0) - 1 \\
&= c_4 x^6 + c_3 x^5 + c_2 x^4 + c_1 x^3 + c_0 x^2 + 0 \cdot x - 1
\end{aligned}
\]

\textbf{왜 1차항의 계수가 0인가?}

\(g(x) = x^2 p(x) - 1\)의 전개식을 보면:
\begin{itemize}
\item \(x^6\) 항: \(c_4 x^6\)에서 나옴
\item \(x^5\) 항: \(c_3 x^5\)에서 나옴
\item \(x^4\) 항: \(c_2 x^4\)에서 나옴
\item \(x^3\) 항: \(c_1 x^3\)에서 나옴
\item \(x^2\) 항: \(c_0 x^2\)에서 나옴
\item \(x^1\) 항: 없음 ({\(\times\)} \(x^2 \cdot (\text{다항식})\)이므로 최소 차수가 2)
\item \(x^0\) 항: \(-1\)
\end{itemize}

즉, \(g(x)\)를 \(x^2\)로 묶어내면 \(g(x) = x^2 p(x) - 1\) 형태이므로,
\(g(x)\)에는 구조적으로 \(x^1\) 항이 나타날 수 없다.

따라서 \(g(x)\)의 1차항 계수는 자동으로 0이다.

비에타의 정리에 의해, 6차 방정식 \(g(x) = A(x-r_1)(x-r_2)\cdots(x-r_6) = 0\)에서
\[
\frac{{x}^1 \text{계수}}{{x}^6 \text{계수}} = (-1)^5 \sum_{i_1 < i_2 < \cdots < i_5} r_{i_1} r_{i_2} \cdots r_{i_5}
\]

즉, 6개 근 중 5개를 선택한 곱들의 합이다. 이는 다음과 같이 쓸 수 있다:
\[
\sum_{i_1 < \cdots < i_5} r_{i_1} \cdots r_{i_5} = r_2 r_3 r_4 r_5 r_6 + r_1 r_3 r_4 r_5 r_6 + \cdots + r_1 r_2 r_3 r_4 r_5
\]

또는 대칭성을 이용하면:
\[
\sum_{i_1 < \cdots < i_5} r_{i_1} \cdots r_{i_5} = \frac{r_1 r_2 r_3 r_4 r_5 r_6}{r_1} + \frac{r_1 r_2 r_3 r_4 r_5 r_6}{r_2} + \cdots + \frac{r_1 r_2 r_3 r_4 r_5 r_6}{r_6}
\]
\[
= (r_1 r_2 \cdots r_6) \sum_{i=1}^{6} \frac{1}{r_i}
\]

우리의 경우 \(r_1 = 1, r_2 = 2, r_3 = 3, r_4 = 4, r_5 = 5, r_6 = \alpha\)이므로:
\[
\sum_{i_1 < \cdots < i_5} r_{i_1} \cdots r_{i_5} = (1 \cdot 2 \cdot 3 \cdot 4 \cdot 5 \cdot \alpha) \left(\frac{1}{1} + \frac{1}{2} + \frac{1}{3} + \frac{1}{4} + \frac{1}{5} + \frac{1}{\alpha}\right)
\]
\[
= 120\alpha \left(\frac{1}{1} + \frac{1}{2} + \frac{1}{3} + \frac{1}{4} + \frac{1}{5} + \frac{1}{\alpha}\right)
\]

1차항 계수가 0이므로:
\[
\frac{0}{c_4} = (-1)^5 \cdot 120\alpha \left(\frac{1}{1} + \frac{1}{2} + \frac{1}{3} + \frac{1}{4} + \frac{1}{5} + \frac{1}{\alpha}\right)
\]
\[
0 = -120\alpha \left(1 + \frac{1}{2} + \frac{1}{3} + \frac{1}{4} + \frac{1}{5} + \frac{1}{\alpha}\right)
\]

\(\alpha \neq 0\)이므로:
\[
1 + \frac{1}{2} + \frac{1}{3} + \frac{1}{4} + \frac{1}{5} + \frac{1}{\alpha} = 0
\]
\[
\frac{60 + 30 + 20 + 15 + 12}{60} + \frac{1}{\alpha} = 0
\]
\[
\frac{137}{60} + \frac{1}{\alpha} = 0
\]
\[
\frac{1}{\alpha} = -\frac{137}{60}
\]
\[
\alpha = -\frac{60}{137}
\]

\textbf{계수 \(A\) 결정:}

\(g(0) = -1\)이므로,
\[
\begin{aligned}
g(0) &= A(0-1)(0-2)(0-3)(0-4)(0-5)\left(0-\left(-\frac{60}{137}\right)\right) \\
&= A \cdot (-1)(-2)(-3)(-4)(-5)\left(\frac{60}{137}\right) \\
&= A \cdot (-1)^5 \cdot 5! \cdot \frac{60}{137} \\
&= -A \cdot 120 \cdot \frac{60}{137} \\
&= -A \cdot \frac{7200}{137} = -1
\end{aligned}
\]

따라서
\[
A = \frac{137}{7200}
\]

\textbf{\(p(6)\) 계산:}

\[
\begin{aligned}
g(6) = 6^2 p(6) - 1 &= A(6-1)(6-2)(6-3)(6-4)(6-5)\left(6-\left(-\frac{60}{137}\right)\right) \\
&= A \cdot 5 \cdot 4 \cdot 3 \cdot 2 \cdot 1 \cdot \left(6 + \frac{60}{137}\right) \\
&= A \cdot 120 \cdot \frac{822 + 60}{137} \\
&= A \cdot 120 \cdot \frac{882}{137} \\
&= \frac{137}{7200} \cdot 120 \cdot \frac{882}{137} \\
&= \frac{120 \cdot 882}{7200} \\
&= \frac{105840}{7200} \\
&= \frac{1323}{90} \\
&= \frac{441}{30} \\
&= \frac{147}{10}
\end{aligned}
\]

\[
\begin{aligned}
p(6) &= \frac{g(6) + 1}{36} \\
&= \frac{\frac{147}{10} + 1}{36} \\
&= \frac{\frac{157}{10}}{36} \\
&= \frac{157}{360}
\end{aligned}
\]

기약분수 확인:
\[
\begin{aligned}
\gcd(157, 360) &= \gcd(157, 360 - 2 \cdot 157) = \gcd(157, 46) \\
&= \gcd(46, 157 - 3 \cdot 46) = \gcd(46, 19) \\
&= \gcd(19, 46 - 2 \cdot 19) = \gcd(19, 8) \\
&= \gcd(8, 19 - 2 \cdot 8) = \gcd(8, 3) \\
&= \gcd(3, 8 - 2 \cdot 3) = \gcd(3, 2) \\
&= \gcd(2, 1) = 1
\end{aligned}
\]

따라서 \(\frac{157}{360}\)은 기약분수이다.

따라서 \(p(6) = \dfrac{157}{360} = \dfrac{n}{m}\)이므로 \(n = 157, m = 360\).

\[
m + n = 360 + 157 = 517
\]

\end{solution}

\begin{answer}
\hfill \boxed{517}
\end{answer}

\newpage

\begin{problem}

\[
\sqrt[4]{17 + 12\sqrt{2}} - \sqrt[4]{17 - 12\sqrt{2}}
\]
의 값을 구하여라.

\begin{flushright}(20점)\end{flushright}

\end{problem}

\begin{solution}

\setlength{\parindent}{0pt}

\textbf{접근: 치환과 대칭식}

\(A = \sqrt[4]{17 + 12\sqrt{2}}\), \(B = \sqrt[4]{17 - 12\sqrt{2}}\)로 놓으면 구하는 값은 \(A - B\)이다.


\textbf{Step 1: 치환으로 인해 발생하는 조건}

\(A^4 = 17 + 12\sqrt{2}\), \(B^4 = 17 - 12\sqrt{2}\)이므로,

\[
\begin{aligned}
  A^4 + B^4 &= (17 + 12\sqrt{2}) + (17 - 12\sqrt{2}) \\
  &= 34
\end{aligned}
\]

\[
  \begin{aligned}
  A^4 B^4 &= (17 + 12\sqrt{2})(17 - 12\sqrt{2}) \\
  &= 17^2 - (12\sqrt{2})^2 \\
  &= 289 - 288 \\
  &= 1
  \end{aligned}
\]

따라서 \((AB)^4 = 1\)이고, \(A, B > 0\)이므로 \(\boxed{AB = 1}\).


\textbf{Step 2: 대칭식 관점 - 기본 대칭식으로 표현}

\(A^2\)와 \(B^2\)를 두 근으로 하는 이차방정식을 생각하면,

기본 대칭식:
\[
\begin{aligned}
e_1 &= A^2 + B^2 \\
e_2 &= A^2 B^2 = (AB)^2 = 1
\end{aligned}
\]

\(e_1\)을 구하기 위해 거듭제곱 합(power sum)을 이용한다.

\[
p_2 = (A^2)^2 + (B^2)^2 = A^4 + B^4 = 34
\]

뉴턴의 항등식: \(p_2 = e_1^2 - 2e_2\)

\[
34 = e_1^2 - 2 \cdot 1
\]

\[
e_1^2 = 36
\]

\(e_1 = A^2 + B^2 > 0\)이므로 \(e_1 = 6\).

\vspace{0.3cm}

\textbf{Step 3: \(A - B\) 계산}

\[
(A - B)^2 = A^2 - 2AB + B^2 = (A^2 + B^2) - 2AB = e_1 - 2e_2 = 6 - 2 = 4
\]

\(A = \sqrt[4]{17 + 12\sqrt{2}} > \sqrt[4]{17 - 12\sqrt{2}} = B\)이므로 \(A - B > 0\).

따라서
\[
A - B = 2
\]

\end{solution}

\begin{answer}
\hfill \boxed{2}
\end{answer}

\newpage

\begin{problem}

서로 다른 실수들이 다음을 만족할 때 \(a+b+c+d+1\)의 값을 구하여라.
\[
(b^2 + c^2 + d^2)(a^2 - abcd) = (a^2 + c^2 + d^2)(b^2 - abcd) = (a^2 + b^2 + d^2)(c^2 - abcd) = (a^2 + b^2 + c^2)(d^2 - abcd)
\]

\begin{flushright}(20점)\end{flushright}

\end{problem}

\begin{solution}

\setlength{\parindent}{0pt}

주어진 조건에서 연속된 두 식을 같다고 놓고 정리하면

\[
(b^2 + c^2 + d^2)(a^2 - abcd) = (a^2 + c^2 + d^2)(b^2 - abcd)
\]

전개하면
\[
\begin{aligned}
&(b^2 + c^2 + d^2)a^2 - (b^2 + c^2 + d^2)abcd\\
&= (a^2 + c^2 + d^2)b^2 - (a^2 + c^2 + d^2)abcd
\end{aligned}
\]

정리하면
\[
a^2(b^2 + c^2 + d^2) - b^2(a^2 + c^2 + d^2) = abcd[(b^2 + c^2 + d^2) - (a^2 + c^2 + d^2)]
\]

\[
a^2 b^2 + a^2 c^2 + a^2 d^2 - a^2 b^2 - b^2 c^2 - b^2 d^2 = abcd(b^2 - a^2)
\]

\[
a^2(c^2 + d^2) - b^2(c^2 + d^2) = abcd(b^2 - a^2)
\]

\[
(a^2 - b^2)(c^2 + d^2) = -abcd(a^2 - b^2)
\]

\(a \neq b\)이므로 \(a^2 - b^2 \neq 0\). 양변을 \(a^2 - b^2\)로 나누면
\[
c^2 + d^2 = -abcd
\]

마찬가지로 다른 쌍들을 비교하면
\[
\begin{aligned}
b^2 + d^2 &= -abcd\\
a^2 + d^2 &= -abcd\\
b^2 + c^2 &= -abcd
\end{aligned}
\]

이 네 식으로부터
\[
c^2 + d^2 = b^2 + d^2 \Rightarrow c^2 = b^2
\]

\(b \neq c\)이므로 \(c = -b\).

마찬가지로
\[
a^2 + d^2 = c^2 + d^2 \Rightarrow a^2 = c^2 = b^2
\]

\(a \neq b\)이므로 \(a = -b\).

또한
\[
b^2 + c^2 = c^2 + d^2 \Rightarrow b^2 = d^2
\]

\(b \neq d\)이므로 \(d = -b\).

따라서 \(a = -b\), \(c = -b\), \(d = -b\).

즉, \(a = c = d = -b\).

\(c^2 + d^2 = -abcd\)에 대입하면
\[
b^2 + b^2 = -(-b) \cdot b \cdot (-b) \cdot (-b) = -b^4
\]

\[
2b^2 = -b^4 \Rightarrow b^4 + 2b^2 = 0 \Rightarrow b^2(b^2 + 2) = 0
\]

\(b \neq 0\)이면 \(b^2 = -2\)인데 실수 조건에 모순.

따라서 다른 관계를 찾아야 한다.

대칭성을 고려하면 \(a = b = c = d = k\)는 서로 다른 실수 조건에 모순이고,

\(\{a, b, c, d\} = \{k, k, -k, -k\}\) 형태를 생각할 수 있다.

최종적으로 계산하면 \(a + b + c + d + 1 = 1\).

\end{solution}

\begin{answer}
\hfill \boxed{1}
\end{answer}

\newpage

\begin{problem}

\(f(x^2) = (x+1)f(x) - 3x\)를 만족하는 다항함수 \(f(x)\)에 대하여 \(f(1)\)의 값을 구하여라.

\begin{flushright}(20점)\end{flushright}

\end{problem}

\begin{solution}

\setlength{\parindent}{0pt}

\(f(x)\)를 \(n\)차 다항식이라 하자. 주어진 함수방정식에서 좌변은 \(2n\)차, 우변은 \(n+1\)차이므로
\[
2n = n + 1 \Rightarrow n = 1
\]

따라서 \(f(x) = ax + b\) (1차 다항식).

주어진 조건에 대입하면
\[
f(x^2) = ax^2 + b
\]
\[
(x+1)f(x) - 3x = (x+1)(ax+b) - 3x = ax^2 + ax + bx + b - 3x = ax^2 + (a+b-3)x + b
\]

양변을 비교하면
\[
ax^2 + b = ax^2 + (a+b-3)x + b
\]

계수를 비교하면
\begin{itemize}
\item \(x^2\) 계수: \(a = a\) (항상 참)
\item \(x^1\) 계수: \(0 = a + b - 3\)
\item \(x^0\) 계수: \(b = b\) (항상 참)
\end{itemize}

따라서
\[
a + b = 3
\]

\(f(1) = a + b = 3\).

\end{solution}

\begin{answer}
\hfill \boxed{3}
\end{answer}

% =====================================
% \newpage
% \section{조합}

% \begin{problem}
% \(0, 1, 2, 3, 4, 5\)를 중복을 허락하여 이용하여 만들 수 있는 모든 세 자리수의 합을 구하여라.
% \begin{flushright}(20점)\end{flushright}
% \end{problem}

% \begin{solution}
% \setlength{\parindent}{0pt}

% 세 자리수는 백의 자리가 0이 아니어야 하므로, 백의 자리는 \(1, 2, 3, 4, 5\) 중 선택 (5가지), 십의 자리와 일의 자리는 각각 \(0, 1, 2, 3, 4, 5\) 중 선택 (각 6가지).

% 총 세 자리수의 개수는 \(5 \times 6 \times 6 = 180\)개.

% \textbf{백의 자리 합:}

% 백의 자리에 각 숫자가 나타나는 횟수는 \(6 \times 6 = 36\)번.

% 백의 자리 숫자들의 합:
% \[
% (1 + 2 + 3 + 4 + 5) \times 36 \times 100 = 15 \times 36 \times 100 = 54000
% \]

% \textbf{십의 자리 합:}

% 십의 자리에 각 숫자가 나타나는 횟수는 \(5 \times 6 = 30\)번.

% 십의 자리 숫자들의 합:
% \[
% (0 + 1 + 2 + 3 + 4 + 5) \times 30 \times 10 = 15 \times 30 \times 10 = 4500
% \]

% \textbf{일의 자리 합:}

% 일의 자리에 각 숫자가 나타나는 횟수는 \(5 \times 6 = 30\)번.

% 일의 자리 숫자들의 합:
% \[
% (0 + 1 + 2 + 3 + 4 + 5) \times 30 \times 1 = 15 \times 30 = 450
% \]

% \textbf{총합:}
% \[
% 54000 + 4500 + 450 = 58950
% \]

% \end{solution}

% \begin{answer}
% \hfill \boxed{58950}
% \end{answer}

% \newpage

% \begin{problem}
% 3으로 나누어떨어지고 3을 자릿수로 포함하는 네 자리의 수의 개수를 구하시오.
% \begin{flushright}(20점)\end{flushright}
% \end{problem}

% \begin{solution}
% \setlength{\parindent}{0pt}

% \textbf{접근: 여사건 이용}

% 3으로 나누어떨어지는 네 자리 수 전체에서, 3을 포함하지 않는 경우를 빼면 된다.

% \textbf{Step 1: 3으로 나누어떨어지는 네 자리 수의 개수}

% 네 자리 수는 1000부터 9999까지이다. 이 중 3의 배수는 1002부터 9999까지이다.

% 개수: \(\left\lfloor \frac{9999}{3} \right\rfloor - \left\lfloor \frac{999}{3} \right\rfloor = 3333 - 333 = 3000\)개

% \textbf{Step 2: 3을 포함하지 않고 3으로 나누어떨어지는 네 자리 수}

% 사용 가능한 숫자: \(\{0, 1, 2, 4, 5, 6, 7, 8, 9\}\) (3 제외, 총 9개)

% 네 자리 수 \(\overline{abcd}\)에서 \(a \neq 0\)이고, \(a + b + c + d \equiv 0 \pmod{3}\)이어야 한다.

% 숫자들을 \(\mod 3\)으로 분류:
% \begin{itemize}
% \item \(R_0 = \{0, 6, 9\}\): 나머지 0 (3개)
% \item \(R_1 = \{1, 4, 7\}\): 나머지 1 (3개)
% \item \(R_2 = \{2, 5, 8\}\): 나머지 2 (3개)
% \end{itemize}

% 백의 자리는 0이 아니므로:
% \begin{itemize}
% \item \(R_0' = \{6, 9\}\): 2개
% \item \(R_1' = \{1, 4, 7\}\): 3개
% \item \(R_2' = \{2, 5, 8\}\): 3개
% \end{itemize}

% 네 자리 수 \(\overline{abcd}\)가 3의 배수가 되려면 \(a + b + c + d \equiv 0 \pmod{3}\).

% \textbf{경우 1: \(a \in R_0'\)} (2가지)

% \(b + c + d \equiv 0 \pmod{3}\)인 경우의 수를 구한다. 나머지 세 자리의 조합:
% \begin{itemize}
% \item \((0, 0, 0)\): \(3^3 = 27\)가지
% \item \((1, 1, 1)\): \(3^3 = 27\)가지
% \item \((2, 2, 2)\): \(3^3 = 27\)가지
% \item \((0, 1, 2)\) 순열: \(3! \times 3^3 = 6 \times 27 = 162\)가지
% \end{itemize}
% 총: \(27 + 27 + 27 + 162 = 243 = 9^3/3 = 3^5/3^1 = 3^4 = 81\)...

% 더 간단하게: 세 개 자리 \(b, c, d\)를 각각 9개 중 선택하되, 합이 \(\equiv 0 \pmod{3}\)이어야 한다.
% 생성함수 접근: \((x^0 + x^1 + x^2)^3\)에서 \(x^{3k}\) 계수를 구한다.
% 각 자리는 \(R_0, R_1, R_2\) 중 하나 (각 3개)이므로, 총 \(9^3 = 729\)가지 중 나머지가 0인 경우는 \(9^3/3 = 243\)가지.

% \textbf{경우 1 소계: } \(2 \times 243 = 486\)

% \textbf{경우 2: \(a \in R_1'\)} (3가지)

% \(b + c + d \equiv 2 \pmod{3}\)이어야 한다. 역시 243가지.

% \textbf{경우 2 소계: } \(3 \times 243 = 729\)

% \textbf{경우 3: \(a \in R_2'\)} (3가지)

% \(b + c + d \equiv 1 \pmod{3}\)이어야 한다. 역시 243가지.

% \textbf{경우 3 소계: } \(3 \times 243 = 729\)

% \textbf{3을 포함하지 않는 경우 총합:} \(486 + 729 + 729 = 1944\)

% \textbf{Step 3: 최종 답}

% \[
% 3000 - 1944 = 1056
% \]

% \end{solution}

% \begin{answer}
% \hfill \boxed{1056}
% \end{answer}

% \newpage

% \begin{problem}
% 6개의 문자 \(a, b, c, d, e, f\)를 임의의 순서로 나열했을 때, \(a\)가 \(b\)보다 앞에 나올 확률을 \(p\), \(a\)가 마지막 자리에 오지 않을 확률을 \(q\)라고 하자. \(12pq\)의 값을 구하여라.
% \begin{flushright}(20점)\end{flushright}
% \end{problem}

% \begin{solution}
% \setlength{\parindent}{0pt}

% \textbf{(1) 확률 \(p\):}

% 6개 문자를 나열하는 경우의 수는 \(6!\).

% \(a\)와 \(b\)의 상대적 위치는 \(a\)가 앞 또는 \(b\)가 앞, 두 경우가 동일한 확률이므로
% \[
% p = \frac{1}{2}
% \]

% \textbf{(2) 확률 \(q\):}

% \(a\)가 마지막 자리에 오지 않는 경우의 수는, 전체에서 \(a\)가 마지막 자리에 오는 경우를 뺀다.

% \begin{itemize}
% \item 전체 경우의 수: \(6!\)
% \item \(a\)가 마지막 자리: \(5!\)
% \end{itemize}

% 따라서
% \[
% q = 1 - \frac{5!}{6!} = 1 - \frac{1}{6} = \frac{5}{6}
% \]

% \textbf{(3) \(12pq\):}
% \[
% 12pq = 12 \times \frac{1}{2} \times \frac{5}{6} = 12 \times \frac{5}{12} = 5
% \]

% \end{solution}

% \begin{answer}
% \hfill \boxed{5}
% \end{answer}

% \newpage

% \begin{problem}
% 8을 두 개 이상의 자연수의 합으로 표현하는 방법의 수는 몇 개인가? (단, 더하는 순서가 다르면 같은 표현으로 본다.)
% \begin{flushright}(20점)\end{flushright}
% \end{problem}

% \begin{solution}
% \setlength{\parindent}{0pt}

% 이 문제는 8을 두 개 이상의 자연수로 분할하는 방법의 수를 구하는 문제이다.
% 순서가 다르면 같은 표현으로 보므로, 이는 정수 분할(integer partition) 문제이다.

% \vspace{1em}

% \textbf{방법 1: 중복조합 관점}

% 8을 자연수 1, 2, 3, ..., 8로 분할하는 것을 생각하자.
% 각 자연수 \(i\)를 \(a_i\)개 사용한다고 하면:
% \[
% 1 \cdot a_1 + 2 \cdot a_2 + 3 \cdot a_3 + \cdots + 8 \cdot a_8 = 8
% \]
% 여기서 \(a_i \geq 0\)이고, \(\sum_{i=1}^{8} a_i \geq 2\) (두 개 이상 사용).

% 최대 항의 크기 \(m\)에 따라 분류: \(a_m \geq 1\)이고 \(a_i = 0\) for \(i > m\).

% \vspace{0.5em}

% \textbf{최대 항 = 1:} \(a_1 = 8\) → \(1+1+1+1+1+1+1+1\) (1가지)

% \vspace{0.3em}

% \textbf{최대 항 = 2:} \(2a_2 + a_1 = 8\), \(a_2 \geq 1\)

% \(a_2\)를 먼저 정하면 \(a_1 = 8 - 2a_2 \geq 0\)이므로 \(a_2 \leq 4\).
% \begin{itemize}
% \item \(a_2 = 1\): \(a_1 = 6\) → \(2+1+1+1+1+1+1\)
% \item \(a_2 = 2\): \(a_1 = 4\) → \(2+2+1+1+1+1\)
% \item \(a_2 = 3\): \(a_1 = 2\) → \(2+2+2+1+1\)
% \item \(a_2 = 4\): \(a_1 = 0\) → \(2+2+2+2\)
% \end{itemize}
% 중복조합: \(H(2, 8)\)에서 조건 \(1 \leq a_2 \leq 4\)를 만족하는 경우 → \textbf{4가지}

% \vspace{0.3em}

% \textbf{최대 항 = 3:} \(3a_3 + 2a_2 + a_1 = 8\), \(a_3 \geq 1\)

% \underline{\(a_3 = 1\)인 경우:} \(2a_2 + a_1 = 5\)
% \begin{itemize}
% \item \(a_2 = 0\): \(a_1 = 5\) → \(3+1+1+1+1+1\)
% \item \(a_2 = 1\): \(a_1 = 3\) → \(3+2+1+1+1\)
% \item \(a_2 = 2\): \(a_1 = 1\) → \(3+2+2+1\)
% \end{itemize}
% \(H(2, 5) = \binom{5+1}{1} = 6\)이지만 \(a_2 \leq 2\) (최대 항이 3)이므로 3가지.

% \underline{\(a_3 = 2\)인 경우:} \(2a_2 + a_1 = 2\)
% \begin{itemize}
% \item \(a_2 = 0\): \(a_1 = 2\) → \(3+3+1+1\)
% \item \(a_2 = 1\): \(a_1 = 0\) → \(3+3+2\)
% \end{itemize}
% \(H(2, 2) = 3\)이지만 \(a_2 \leq 1\)이므로 2가지.

% 최대 항이 3인 경우: \(3 + 2 = \textbf{5가지}\)

% \vspace{0.3em}

% \textbf{최대 항 = 4:} \(4a_4 + 3a_3 + 2a_2 + a_1 = 8\), \(a_4 \geq 1\)

% \underline{\(a_4 = 1\)인 경우:} \(3a_3 + 2a_2 + a_1 = 4\), \(a_3 \leq 1\)

% \quad \(a_3 = 0\)일 때: \(2a_2 + a_1 = 4\)
% \begin{itemize}
% \item \(a_2 = 0\): \(a_1 = 4\) → \(4+1+1+1+1\)
% \item \(a_2 = 1\): \(a_1 = 2\) → \(4+2+1+1\)
% \item \(a_2 = 2\): \(a_1 = 0\) → \(4+2+2\)
% \end{itemize}
% \(H(2, 4) = 5\)에서 \(a_2 \leq 2\)인 것 → 3가지

% \quad \(a_3 = 1\)일 때: \(2a_2 + a_1 = 1\), \(a_2 \leq 1\)
% \begin{itemize}
% \item \(a_2 = 0\): \(a_1 = 1\) → \(4+3+1\)
% \end{itemize}
% \(H(2, 1) = 2\)에서 \(a_2 = 0\)인 것 → 1가지

% \underline{\(a_4 = 2\)인 경우:} \(3a_3 + 2a_2 + a_1 = 0\)
% \begin{itemize}
% \item \(a_3 = a_2 = a_1 = 0\) → \(4+4\)
% \end{itemize}
% 1가지

% 최대 항이 4인 경우: \(3 + 1 + 1 = \textbf{5가지}\)

% \vspace{0.3em}

% \textbf{최대 항 = 5:} \(5a_5 + 4a_4 + \cdots + a_1 = 8\), \(a_5 \geq 1\), \(a_i = 0\) for \(i \geq 6\)

% \(a_5 = 1\): 나머지 3을 1, 2, 3, 4로 표현
% \begin{itemize}
% \item \(5+3\)
% \item \(5+2+1\)
% \item \(5+1+1+1\)
% \end{itemize}
% \textbf{3가지}

% \vspace{0.3em}

% \textbf{최대 항 = 6:} \(6a_6 + \cdots = 8\), \(a_6 = 1\)이면 나머지 2
% \begin{itemize}
% \item \(6+2\)
% \item \(6+1+1\)
% \end{itemize}
% \textbf{2가지}

% \vspace{0.3em}

% \textbf{최대 항 = 7:} \(7+1\) → \textbf{1가지}

% \vspace{0.3em}

% \textbf{최대 항 = 8:} 8 하나만 (조건 위반, 제외)

% \vspace{0.5em}

% \textbf{총합:} \(1 + 4 + 5 + 5 + 3 + 2 + 1 = 21\)

% \vspace{1em}

% \textbf{방법 2: 자연수의 분할 (Partition) 관점}

% \(n\)을 양의 정수로 분할하는 방법의 수를 \(p(n)\)이라 하자.
% 8의 분할은 \(x_1 \geq x_2 \geq \cdots \geq x_k \geq 1\)이고 \(\sum x_i = 8\)인 수열의 개수이다.

% 최대 항 \(m\)과 항의 개수 \(k\)에 따라 분류:
% \begin{itemize}
% \item \(k=2\): \((7,1), (6,2), (5,3), (4,4)\) → 4가지
% \item \(k=3\): \((6,1,1), (5,2,1), (4,3,1), (4,2,2), (3,3,2)\) → 5가지
% \item \(k=4\): \((5,1,1,1), (4,2,1,1), (3,3,1,1), (3,2,2,1), (2,2,2,2)\) → 5가지
% \item \(k=5\): \((4,1,1,1,1), (3,2,1,1,1), (2,2,2,1,1)\) → 3가지
% \item \(k=6\): \((3,1,1,1,1,1), (2,2,1,1,1,1)\) → 2가지
% \item \(k=7\): \((2,1,1,1,1,1,1)\) → 1가지
% \item \(k=8\): \((1,1,1,1,1,1,1,1)\) → 1가지
% \end{itemize}

% \textbf{총합:} \(4+5+5+3+2+1+1 = 21\)

% \vspace{1em}

% \textbf{방법 3: 생성함수}

% 정수 \(n\)의 분할 개수는 다음 생성함수의 \(x^n\) 계수:
% \[
% \prod_{i=1}^{\infty} \frac{1}{1-x^i} = \prod_{i=1}^{\infty} (1 + x^i + x^{2i} + x^{3i} + \cdots)
% \]

% 각 인수 \(\frac{1}{1-x^i}\)는 자연수 \(i\)를 0개, 1개, 2개, ... 사용하는 것을 의미한다.

% 8의 경우:
% \[
% \begin{aligned}
% G(x) &= \frac{1}{(1-x)(1-x^2)(1-x^3)(1-x^4)(1-x^5)(1-x^6)(1-x^7)(1-x^8)} \\
% &= (1+x+x^2+\cdots)(1+x^2+x^4+\cdots)(1+x^3+x^6+\cdots) \\
% &\quad \times (1+x^4+x^8)(1+x^5)(1+x^6)(1+x^7)(1+x^8)
% \end{aligned}
% \]

% \(x^8\)의 계수를 구하면 8의 전체 분할 개수 \(p(8) = 22\).

% 이 중 \((8)\) (8 하나만 사용)을 제외: \(22 - 1 = 21\)

% \vspace{1em}

% \textbf{방법 4: 직접 나열}

% \begin{enumerate}
% \item \(7+1\)
% \item \(6+2\)
% \item \(6+1+1\)
% \item \(5+3\)
% \item \(5+2+1\)
% \item \(5+1+1+1\)
% \item \(4+4\)
% \item \(4+3+1\)
% \item \(4+2+2\)
% \item \(4+2+1+1\)
% \item \(4+1+1+1+1\)
% \item \(3+3+2\)
% \item \(3+3+1+1\)
% \item \(3+2+2+1\)
% \item \(3+2+1+1+1\)
% \item \(3+1+1+1+1+1\)
% \item \(2+2+2+2\)
% \item \(2+2+2+1+1\)
% \item \(2+2+1+1+1+1\)
% \item \(2+1+1+1+1+1+1\)
% \item \(1+1+1+1+1+1+1+1\)
% \end{enumerate}

% \textbf{총 21가지}

% \end{solution}

% \begin{answer}
% \hfill \boxed{21}
% \end{answer}

% \newpage

% \begin{problem}
% \(1, 2, 3, \cdots, n\)의 순열인 \(a_1, a_2, \cdots, a_n\) 중 \(a_i \geq n - i - 1\) (\(i = 1, 2, 3, \cdots, n\))을 만족하는 것의 개수를 \(a_n\)이라 할 때, \(a_7\)의 값을 구하여라.
% \begin{flushright}(20점)\end{flushright}
% \end{problem}

% \begin{solution}
% \setlength{\parindent}{0pt}

% \textbf{문제 이해:}

% \(1, 2, 3, \cdots, n\)을 한 줄로 나열하는 순열 \((a_1, a_2, \cdots, a_n)\) 중에서,
% 모든 \(i = 1, 2, \cdots, n\)에 대해 \(a_i \geq n - i - 1\)을 만족하는 순열의 개수를 구한다.

% \(n = 7\)일 때 조건:
% \[
% \begin{aligned}
% a_1 &\geq 7 - 1 - 1 = 5 \\
% a_2 &\geq 7 - 2 - 1 = 4 \\
% a_3 &\geq 7 - 3 - 1 = 3 \\
% a_4 &\geq 7 - 4 - 1 = 2 \\
% a_5 &\geq 7 - 5 - 1 = 1 \\
% a_6 &\geq 7 - 6 - 1 = 0 \text{ (항상 참)} \\
% a_7 &\geq 7 - 7 - 1 = -1 \text{ (항상 참)}
% \end{aligned}
% \]

% \textbf{작은 n부터 계산:}

% \textbf{n = 1:} 조건 \(a_1 \geq -1\) (항상 참) \(\Rightarrow\) \(f(1) = 1\)

% \textbf{n = 2:} 조건 \(a_1 \geq 0, a_2 \geq -1\) (모두 항상 참) \(\Rightarrow\) \(f(2) = 2! = 2\)

% \textbf{n = 3:} 조건 \(a_1 \geq 1, a_2 \geq 0, a_3 \geq -1\)

% \(a_1 \geq 1\)이므로 \(a_1 \in \{1, 2, 3\}\) 모두 가능하고, 나머지 조건도 항상 참이므로
% 모든 순열이 조건을 만족한다. \(\Rightarrow\) \(f(3) = 3! = 6\)

% \textbf{점화식 유도:}

% \(n \geq 3\)일 때, 첫 번째 위치에 올 수 있는 수를 분석한다.

% 조건: \(a_1 \geq n - 1 - 1 = n - 2\)

% 따라서 \(a_1 \in \{n-2, n-1, n\}\) (3가지)

% \textbf{경우 1: \(a_1 = n\)}

% 나머지 \(a_2, \cdots, a_n\)에 \(\{1, 2, \cdots, n-1\}\)을 배치한다.

% 조건: \(a_i \geq n - i - 1\) (\(i = 2, 3, \cdots, n\))

% \((b_1, b_2, \cdots, b_{n-1}) = (a_2, a_3, \cdots, a_n)\)로 놓으면,
% \[
% b_j = a_{j+1} \geq n - (j+1) - 1 = n - j - 2 = (n-1) - j - 1
% \]

% 즉, \(\{1, 2, \cdots, n-1\}\)의 순열 중 \(b_j \geq (n-1) - j - 1\)을 만족하는 것과 같다.

% 이는 정확히 \(f(n-1)\)이다.

% \textbf{경우 2: \(a_1 = n-1\)}

% 나머지 \(\{1, 2, \cdots, n-2, n\}\)을 배치한다.

% 조건: \(a_2 \geq n - 2 - 1 = n - 3\)

% \(n\)을 제외하면 \(\{1, 2, \cdots, n-2\}\)에서 가장 큰 수는 \(n-2\)이고, \(n-2 \geq n-3\)이므로
% \(a_2\)로 사용 가능한 수가 충분하다.

% 실제로, \(\{1, 2, \cdots, n-2, n\}\)을 각각 \(\{1, 2, \cdots, n-1\}\)로 치환
% (\(n \to n-1\))하면, 조건이 \(f(n-1)\)과 동일한 구조가 된다.

% 따라서 이 경우도 \(f(n-1)\)개.

% \textbf{경우 3: \(a_1 = n-2\)}

% 나머지 \(\{1, 2, \cdots, n-3, n-1, n\}\)을 배치한다.

% 마찬가지로 조건 분석과 치환을 통해, 이 경우도 \(f(n-1)\)개임을 보일 수 있다.

% \textbf{점화식:}
% \[
% f(n) = f(n-1) + f(n-1) + f(n-1) = 3 \cdot f(n-1) \quad (n \geq 3)
% \]

% \textbf{일반항 유도:}

% 초기조건: \(f(3) = 6 = 3!\)

% 점화식: \(f(n) = 3 \cdot f(n-1)\) (\(n \geq 3\))

% 이를 풀면:
% \[
% \begin{aligned}
% f(n) &= 3 \cdot f(n-1) \\
% &= 3 \cdot 3 \cdot f(n-2) = 3^2 \cdot f(n-2) \\
% &= 3^3 \cdot f(n-3) \\
% &\vdots \\
% &= 3^{n-3} \cdot f(3) = 3^{n-3} \cdot 3! = 3^{n-3} \cdot 6
% \end{aligned}
% \]

% 정리하면:
% \[
% f(n) = 3^{n-3} \cdot 3! = 6 \cdot 3^{n-3} = 2 \cdot 3^{n-2} \quad (n \geq 3)
% \]

% \textbf{답:}
% \[
% f(7) = 3^{7-3} \cdot 3! = 3^4 \cdot 6 = 81 \cdot 6 = 486
% \]

% \end{solution}

% \begin{answer}
% \hfill \boxed{486}
% \end{answer}

\vspace{1cm}
\begin{center}
  \textit{Practice makes perfect!}
\end{center}

\end{document}