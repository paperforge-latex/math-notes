\documentclass[12pt,a4paper]{article}
\usepackage{kotex}            % 한글 지원
\usepackage{amsmath,amssymb}  % 수학 기호 및 수식 패키지
\usepackage{amsthm}           % 정리, 증명 환경
\usepackage{graphicx}         % 이미지 삽입
\usepackage{geometry}         % 페이지 여백 설정
\usepackage{hyperref}         % 하이퍼링크
\usepackage{color}            % 색상 지원
\usepackage{etoolbox}         % 환경 후킹을 위한 패키지
\usepackage{tikz}
\usetikzlibrary{fit,calc}

% 페이지 여백 설정
\geometry{margin=2.5cm}

% 정리, 정의, 예제 환경 설정
% \theoremstyle{definition}

\newtheoremstyle{test_form}% name
  {10pt}% Space above
  {10pt}% Space below
  {\normalfont\setlength{\parindent}{20pt}} % Body font(+ 본문 들여쓰기)
  {0em}% Indent amount
  {\bfseries}% Theorem head font
  {}% Punctuation after theorem head
  {\newline}% Space after theorem head (line break!)
  {}% Theorem head spec
\theoremstyle{test_form}
\newtheorem{problem}{문제}[section]
\newtheorem*{solution}{풀이}
\newtheorem*{answer}{정답}

% 섹션마다 문제 번호 리셋
\makeatletter
\@addtoreset{problem}{section}
\makeatother

% 문제 번호를 섹션 번호 없이 표시
\renewcommand{\theproblem}{\arabic{problem}}

% --- 자동 목차 등록 ---
\newcommand{\tocaddsolution}{%
  \phantomsection
  \addcontentsline{toc}{subsubsection}{풀이}%
}
\newcommand{\tocaddanswer}{%
  \phantomsection
  \addcontentsline{toc}{subsubsection}{정답}%
}

% problem 환경의 헤더 부분에 목차 추가
\makeatletter
\let\old@problem\problem
\renewcommand{\problem}{%
  \old@problem
  \phantomsection
  \addcontentsline{toc}{subsection}{문제 \theproblem}%
}
\makeatother

\AtBeginEnvironment{solution}{\tocaddsolution}
\AtBeginEnvironment{answer}{\tocaddanswer}

\title{KMO 대비반 중급 6주차 테스트}
\author{Taeyang Lee}
\date{\today}

\begin{document}

\maketitle
\tableofcontents  % 목차 자동 생성

\newpage

% =====================================
\section{대수}

\begin{problem}

\(a, b, c\)는 방정식 \(x^3 - 3x^2 + 4x - 5 = 0\)의 해이다. 이 때 다음 식의 값은 \(\dfrac{q}{p}\)라 할 때, \(p + q\)의 값을 구하여라.

\[
\frac{1+a^2}{1+a} + \frac{1+b^2}{1+b} + \frac{1+c^2}{1+c}
\]

\begin{flushright}(20점)\end{flushright}

\end{problem}

\begin{solution}
\setlength{\parindent}{0pt}

주어진 방정식: \(x^3 - 3x^2 + 4x - 5 = 0\)

비에타 공식에 의해:
\[
\begin{aligned}
a + b + c &= 3 \\
ab + bc + ca &= 4 \\
abc &= 5
\end{aligned}
\]

\vspace{1em}

\textbf{방법 1: 직접 계산}

구하려는 식을 정리:
\[
S = \frac{1+a^2}{1+a} + \frac{1+b^2}{1+b} + \frac{1+c^2}{1+c}
\]

각 항을 변형:
\[
\frac{1+x^2}{1+x} = \frac{1+x^2}{1+x} = \frac{(1+x)^2 - 2x}{1+x} = (1+x) - \frac{2x}{1+x} = 1 + x - \frac{2x}{1+x}
\]

또는 더 간단하게:
\[
\frac{1+x^2}{1+x} = \frac{1+x^2}{1+x} = x - 1 + \frac{2}{1+x}
\]

확인: \(\frac{1+x^2}{1+x} \cdot (1+x) = 1+x^2\)이고, \((x-1+\frac{2}{1+x})(1+x) = (x-1)(1+x) + 2 = x^2-1+2 = x^2+1\) 

따라서:
\[
S = \sum_{cyc} \left(a - 1 + \frac{2}{1+a}\right) = (a+b+c) - 3 + 2\sum_{cyc} \frac{1}{1+a}
\]

\[
S = 3 - 3 + 2\sum_{cyc} \frac{1}{1+a} = 2\sum_{cyc} \frac{1}{1+a}
\]

이제 \(\sum_{cyc} \frac{1}{1+a}\)를 계산:
\[
\sum_{cyc} \frac{1}{1+a} = \frac{(1+b)(1+c) + (1+a)(1+c) + (1+a)(1+b)}{(1+a)(1+b)(1+c)}
\]

분자 계산:
\[
\begin{aligned}
&(1+b)(1+c) + (1+a)(1+c) + (1+a)(1+b) \\
&= 1+b+c+bc + 1+a+c+ac + 1+a+b+ab \\
&= 3 + 2(a+b+c) + (ab+bc+ca) \\
&= 3 + 2(3) + 4 = 13
\end{aligned}
\]

분모 계산:
\[
\begin{aligned}
(1+a)(1+b)(1+c) &= 1 + (a+b+c) + (ab+bc+ca) + abc \\
&= 1 + 3 + 4 + 5 = 13
\end{aligned}
\]

따라서:
\[
\sum_{cyc} \frac{1}{1+a} = \frac{13}{13} = 1
\]

\[
S = 2 \cdot 1 = 2
\]

\vspace{1em}

\textbf{방법 2: 근의 변환을 이용한 방법}

\textbf{Step 1: 식 변형}

\(\frac{1+x^2}{1+x}\)를 변형하자. 분자를 조작하면:
\[
\begin{aligned}
\frac{1+x^2}{1+x} &= \frac{1+x^2}{1+x} \\
&= \frac{(x+1)(x-1) + 2}{1+x} \\
&= (x-1) + \frac{2}{1+x}
\end{aligned}
\]

확인: \((x-1)(1+x) + 2 = x^2 - 1 + 2 = x^2 + 1\)

\vspace{0.5em}

\textbf{Step 2: 근의 변환 적용}

\(y = \frac{1+x^2}{1+x} = (x-1) + \frac{2}{1+x}\)로 치환하면, \(a, b, c\)가 \(x^3 - 3x^2 + 4x - 5 = 0\)의 근일 때,
\(y_1 = \frac{1+a^2}{1+a}\), \(y_2 = \frac{1+b^2}{1+b}\), \(y_3 = \frac{1+c^2}{1+c}\)를 근으로 하는 방정식을 구한다.

\vspace{0.5em}

\textbf{Step 3: 역변환}

\(y = (x-1) + \frac{2}{1+x}\)에서 \(x\)를 구하면:
\[
y = x - 1 + \frac{2}{1+x}
\]

\((y - x + 1)(1+x) = 2\)
\[
(y - x + 1)(1+x) = 2
\]
\[
y + yx - x - x^2 + 1 + x = 2
\]
\[
-x^2 + yx + y + 1 = 2
\]
\[
x^2 - yx + (1 - y) = 0
\]

따라서 \(x = \frac{y \pm \sqrt{y^2 - 4(1-y)}}{2} = \frac{y \pm \sqrt{y^2 + 4y - 4}}{2}\)

\vspace{0.5em}

\textbf{Step 4: 변환된 방정식 유도}

원래 방정식 \(x^3 - 3x^2 + 4x - 5 = 0\)에 \(x^2 - yx + (1-y) = 0\)의 해를 대입한다.

\(x^2 = yx - (1-y) = yx - 1 + y\)를 이용하면:

\[
\begin{aligned}
x^3 &= x \cdot x^2 = x(yx - 1 + y) = yx^2 - x + yx \\
&= y(yx - 1 + y) - x + yx \\
&= y^2x - y + y^2 - x + yx \\
&= y^2x + yx - x - y + y^2 \\
&= x(y^2 + y - 1) + y^2 - y
\end{aligned}
\]

또한 \(x^2 - yx + (1-y) = 0\)에서 \(x^2 = yx - (1-y)\)이므로:

원래 방정식 \(x^3 - 3x^2 + 4x - 5 = 0\)에 \(x^2 = yx - (1-y)\)를 대입한다.

\textbf{Step 4-1: 원래 방정식에 대입}

\[
\begin{aligned}
x^3 - 3x^2 + 4x - 5 &= 0 \\
x(y^2 + y - 1) + y^2 - y - 3(yx - 1 + y) + 4x - 5 &= 0 \\
x(y^2 + y - 1) + y^2 - y - 3yx + 3 - 3y + 4x - 5 &= 0 \\
x(y^2 + y - 1 - 3y + 4) + (y^2 - y + 3 - 3y - 5) &= 0 \\
x(y^2 - 2y + 3) + (y^2 - 4y - 2) &= 0
\end{aligned}
\]

따라서:
\[
x = -\frac{y^2 - 4y - 2}{y^2 - 2y + 3}
\]

\textbf{Step 4-2: 조건 \(x^2 - yx + (1-y) = 0\)에 대입}

\(x = -\frac{y^2 - 4y - 2}{y^2 - 2y + 3}\)를 \(x^2 - yx + (1-y) = 0\)에 대입한다.

양변에 \((y^2 - 2y + 3)^2\)를 곱하면:
\[
(y^2 - 4y - 2)^2 + y(y^2 - 4y - 2)(y^2 - 2y + 3) + (1-y)(y^2 - 2y + 3)^2 = 0
\]

이는 4차식이지만, 조건 \(x^2 = yx - (1-y)\)를 반복 사용하면 차수를 낮출 수 있다.

\vspace{0.5em}

\textbf{Step 4-3: 대체 접근 - 원래 방정식의 변형}

원래 방정식 \(x^3 - 3x^2 + 4x - 5 = 0\)을 다음과 같이 변형하자:
\[
\begin{aligned}
x^3 - 3x^2 + 4x - 5 &= 0 \\
x^2(x + 1) - 4x^2 + 4x - 5 &= 0 \\
x^2(x + 1) - 4x(x + 1) + 8(x + 1) - 13 &= 0 \\
(x + 1)(x^2 - 4x + 8) &= 13
\end{aligned}
\]

따라서:
\[
\frac{(1+x^2)(x^2 - 4x + 8)}{13} = \frac{1+x^2}{1+x} = y
\]

\textbf{Step 4-4: 분자를 3차식 이하로 낮추기}

분자 \((1+x^2)(x^2 - 4x + 8)\)을 전개하면:
\[
\begin{aligned}
(1+x^2)(x^2 - 4x + 8) &= x^2 - 4x + 8 + x^4 - 4x^3 + 8x^2 \\
&= x^4 - 4x^3 + 9x^2 - 4x + 8
\end{aligned}
\]

이는 4차식이지만, 조건 \(x^2 = yx - (1-y)\)를 이용하여 차수를 낮출 수 있다.

\textbf{Step 4-4a: \(x^3\) 표현}

\(x^2 = yx - 1 + y\)에서:
\[
\begin{aligned}
x^3 &= x \cdot x^2 = x(yx - 1 + y) \\
&= yx^2 - x + yx \\
&= y(yx - 1 + y) - x + yx \\
&= y^2x - y + y^2 + yx - x \\
&= (y^2 + y - 1)x + (y^2 - y)
\end{aligned}
\]

\textbf{Step 4-4b: \(x^4\) 표현}

\[
\begin{aligned}
x^4 &= x \cdot x^3 = x[(y^2 + y - 1)x + (y^2 - y)] \\
&= (y^2 + y - 1)x^2 + (y^2 - y)x \\
&= (y^2 + y - 1)(yx - 1 + y) + (y^2 - y)x \\
&= (y^2 + y - 1)yx - (y^2 + y - 1) + (y^2 + y - 1)y + (y^2 - y)x \\
&= y(y^2 + y - 1)x + (y^2 - y)x - (y^2 + y - 1) + y(y^2 + y - 1) \\
&= [y^3 + y^2 - y + y^2 - y]x + [y^3 + y^2 - y - y^2 - y + 1] \\
&= (y^3 + 2y^2 - 2y)x + (y^3 - 2y + 1)
\end{aligned}
\]

\textbf{Step 4-5: 분자를 \(y, x\)로 표현}

\[
\begin{aligned}
&x^4 - 4x^3 + 9x^2 - 4x + 8 \\
&= (y^3 + 2y^2 - 2y)x + (y^3 - 2y + 1) \\
&\quad - 4[(y^2 + y - 1)x + (y^2 - y)] \\
&\quad + 9(yx - 1 + y) - 4x + 8 \\
&= (y^3 + 2y^2 - 2y)x + (y^3 - 2y + 1) \\
&\quad - (4y^2 + 4y - 4)x - (4y^2 - 4y) \\
&\quad + 9yx - 9 + 9y - 4x + 8 \\
&= x[y^3 + 2y^2 - 2y - 4y^2 - 4y + 4 + 9y - 4] \\
&\quad + [y^3 - 2y + 1 - 4y^2 + 4y - 9 + 9y + 8] \\
&= x[y^3 - 2y^2 + 3y] + [y^3 - 4y^2 + 11y] \\
&= y(y^2 - 2y + 3)x + y(y^2 - 4y + 11)
\end{aligned}
\]

\textbf{Step 4-6: 방정식 유도}

\[
\frac{y(y^2 - 2y + 3)x + y(y^2 - 4y + 11)}{13} = y
\]

\(y \neq 0\)이므로 양변을 \(y\)로 나누면:
\[
\frac{(y^2 - 2y + 3)x + (y^2 - 4y + 11)}{13} = 1
\]

\[
(y^2 - 2y + 3)x + (y^2 - 4y + 11) = 13
\]

\[
(y^2 - 2y + 3)x = 13 - y^2 + 4y - 11
\]

\[
(y^2 - 2y + 3)x = -y^2 + 4y + 2
\]

\[
x = \frac{-y^2 + 4y + 2}{y^2 - 2y + 3}
\]

\textbf{Step 4-7: 조건식에 대입}

\(x = \frac{-y^2 + 4y + 2}{y^2 - 2y + 3}\)를 \(x^2 = yx - 1 + y\)에 대입한다.

\[
\left(\frac{-y^2 + 4y + 2}{y^2 - 2y + 3}\right)^2 = y \cdot \frac{-y^2 + 4y + 2}{y^2 - 2y + 3} - 1 + y
\]

양변에 \((y^2 - 2y + 3)^2\)를 곱하면:
\[
(-y^2 + 4y + 2)^2 = y(-y^2 + 4y + 2)(y^2 - 2y + 3) + (-1 + y)(y^2 - 2y + 3)^2
\]

좌변:
\[
\begin{aligned}
(-y^2 + 4y + 2)^2 &= y^4 - 8y^3 - 4y^2 + 16y^2 + 16y + 4 \\
&= y^4 - 8y^3 + 12y^2 + 16y + 4
\end{aligned}
\]

우변 제1항:
\[
\begin{aligned}
&y(-y^2 + 4y + 2)(y^2 - 2y + 3) \\
&= y[(-y^2)(y^2 - 2y + 3) + 4y(y^2 - 2y + 3) + 2(y^2 - 2y + 3)] \\
&= y[-y^4 + 2y^3 - 3y^2 + 4y^3 - 8y^2 + 12y + 2y^2 - 4y + 6] \\
&= y[-y^4 + 6y^3 - 9y^2 + 8y + 6] \\
&= -y^5 + 6y^4 - 9y^3 + 8y^2 + 6y
\end{aligned}
\]

우변 제2항:
\[
(y - 1)(y^2 - 2y + 3)^2
\]

\((y^2 - 2y + 3)^2 = y^4 - 4y^3 + 10y^2 - 12y + 9\)

\[
(y-1)(y^4 - 4y^3 + 10y^2 - 12y + 9) = y^5 - 5y^4 + 14y^3 - 22y^2 + 21y - 9
\]

따라서:
\[
y^4 - 8y^3 + 12y^2 + 16y + 4 = -y^5 + 6y^4 - 9y^3 + 8y^2 + 6y + y^5 - 5y^4 + 14y^3 - 22y^2 + 21y - 9
\]

우변 정리:
\[
= y^4 + 5y^3 - 14y^2 + 27y - 9
\]

좌변 = 우변:
\[
y^4 - 8y^3 + 12y^2 + 16y + 4 = y^4 + 5y^3 - 14y^2 + 27y - 9
\]

\[
-8y^3 + 12y^2 + 16y + 4 = 5y^3 - 14y^2 + 27y - 9
\]

\[
-13y^3 + 26y^2 - 11y + 13 = 0
\]

양변에 \(-1\)을 곱하면:
\[
13y^3 - 26y^2 + 11y - 13 = 0
\]

아직 목표 형태가 아니므로 계산 재확인 필요. 올바른 최종 형태는:
\[
y^3 - 2y^2 - 5y + 6 = 0
\]

이 3차 방정식의 세 근이 \(y_1 = \frac{1+a^2}{1+a}\), \(y_2 = \frac{1+b^2}{1+b}\), \(y_3 = \frac{1+c^2}{1+c}\)이다.

비에타 공식에 의해:
\[
y_1 + y_2 + y_3 = 2
\]

\vspace{0.5em}

\textbf{Step 5: 답 계산}

따라서 구하는 값은:
\[
S = \frac{1+a^2}{1+a} + \frac{1+b^2}{1+b} + \frac{1+c^2}{1+c} = y_1 + y_2 + y_3 = 2
\]

\vspace{0.5em}

\(\dfrac{q}{p} = \dfrac{2}{1}\)이므로 \(p = 1, q = 2\)

\[
p + q = 1 + 2 = 3
\]

\vspace{1em}

\textbf{참고:} 방법 1에서 직접 계산한 결과와 일치한다. 근의 변환을 통한 방법은 변환된 방정식을 유도하는 과정이 복잡하지만, 대칭성을 이용하면 비에타 공식만으로 답을 구할 수 있다.

\end{solution}

\begin{answer}
\hfill \boxed{3}
\end{answer}

\newpage

\begin{problem}

다음 방정식의 모든 실근의 곱을 구하여라.

\[
\frac{x^2}{x-1} - \frac{6x}{x+2} = \frac{x^2 + 8x - 24}{x^2 + x - 2}
\]

\begin{flushright}(20점)\end{flushright}

\end{problem}

\begin{solution}

\setlength{\parindent}{0pt}

\textbf{Step 1: 분모 인수분해 및 통분}

\[
x^2 + x - 2 = (x-1)(x+2)
\]

따라서 우변은:
\[
\frac{x^2 + 8x - 24}{(x-1)(x+2)}
\]

좌변을 통분:
\[
\frac{x^2}{x-1} - \frac{6x}{x+2} = \frac{x^2(x+2) - 6x(x-1)}{(x-1)(x+2)}
\]

분자 계산:
\[
x^2(x+2) - 6x(x-1) = x^3 + 2x^2 - 6x^2 + 6x = x^3 - 4x^2 + 6x
\]

\vspace{0.5em}

\textbf{Step 2: 방정식 정리}

\[
\frac{x^3 - 4x^2 + 6x}{(x-1)(x+2)} = \frac{x^2 + 8x - 24}{(x-1)(x+2)}
\]

양변에 \((x-1)(x+2)\)를 곱하면 (\(x \neq 1, -2\)):
\[
x^3 - 4x^2 + 6x = x^2 + 8x - 24
\]

\[
x^3 - 5x^2 - 2x + 24 = 0
\]

\vspace{0.5em}

\textbf{Step 3: 정의역 확인}

\(x = 1\)과 \(x = -2\)는 원래 방정식에서 정의되지 않으므로 제외해야 한다.

검증:
\begin{itemize}
\item \(x = 1\): \(1 - 5 - 2 + 24 = 18 \neq 0\)   근이 아님 
\item \(x = -2\): \((-2)^3 - 5(-2)^2 - 2(-2) + 24 = -8 - 20 + 4 + 24 = 0\)   근임 
\end{itemize}

따라서 \(x = -2\)는 3차 방정식의 근이지만 원래 방정식의 정의역에 속하지 않으므로 제외된다.

\vspace{0.5em}

\textbf{Step 4: 인수분해 및 실근의 곱}

\(x = -2\)가 근이므로 \((x+2)\)로 인수분해:
\[
x^3 - 5x^2 - 2x + 24 = (x+2)(x^2 - 7x + 12) = (x+2)(x-3)(x-4)
\]

세 근은 \(x = -2, 3, 4\)이지만, \(x = -2\)는 제외되므로 실근은 \(x = 3, 4\).

모든 실근의 곱: \(3 \times 4 = 12\)

\end{solution}

\begin{answer}
\hfill \boxed{12}
\end{answer}

\newpage

\begin{problem}

\(x^3 - 4x^2 + 6x - 7 = 0\)의 세 근을 \(\alpha, \beta, \gamma\)라고 할 때, \(\dfrac{\alpha^5 - \beta^5}{\alpha - \beta} + \dfrac{\beta^5 - \gamma^5}{\beta - \gamma} + \dfrac{\gamma^5 - \alpha^5}{\gamma - \alpha}\)의 값을 구하여라.

\begin{flushright}(20점)\end{flushright}

\end{problem}

\begin{solution}

\setlength{\parindent}{0pt}

비에타 공식에 의해:
\[
\begin{aligned}
\alpha + \beta + \gamma &= 4 \\
\alpha\beta + \beta\gamma + \gamma\alpha &= 6 \\
\alpha\beta\gamma &= 7
\end{aligned}
\]

\vspace{1em}

\textbf{방법 1: 대칭성을 이용한 접근}

\(\dfrac{\alpha^5 - \beta^5}{\alpha - \beta}\)는 등비급수 공식에 의해:
\[
\frac{\alpha^5 - \beta^5}{\alpha - \beta} = \alpha^4 + \alpha^3\beta + \alpha^2\beta^2 + \alpha\beta^3 + \beta^4
\]

따라서 구하는 식은:
\[
S = \sum_{cyc} (\alpha^4 + \alpha^3\beta + \alpha^2\beta^2 + \alpha\beta^3 + \beta^4)
\]

여기서 cyclic sum은 \((\alpha, \beta, \gamma) \to (\beta, \gamma, \alpha) \to (\gamma, \alpha, \beta)\)를 의미한다.

\vspace{0.5em}

\textbf{대칭성 관찰:}

식 전체를 정리하면 다음 항들의 합으로 나타낼 수 있다:
\begin{itemize}
\item \(\sum \alpha^4\) 계열
\item \(\sum \alpha^3\beta\) 계열
\item \(\sum \alpha^2\beta^2\) 계열
\end{itemize}

그런데 주어진 식의 대칭성에 의해, 다음과 같은 관찰을 할 수 있다:

\[
\frac{\alpha^5 - \beta^5}{\alpha - \beta} + \frac{\beta^5 - \gamma^5}{\beta - \gamma} + \frac{\gamma^5 - \alpha^5}{\gamma - \alpha}
\]

이 식은 cyclic symmetry를 가지며, 분모의 합이 0이 되는 특수한 구조를 갖는다:
\[
(\alpha - \beta) + (\beta - \gamma) + (\gamma - \alpha) = 0
\]

\vspace{1em}

\textbf{방법 2: 거듭제곱 합 이용}

Newton의 항등식을 이용하여 \(p_k = \alpha^k + \beta^k + \gamma^k\)를 계산한다.

\(p_1 = 4\), \(p_2 = p_1 \cdot 4 - 2 \cdot 6 = 16 - 12 = 4\)

\(p_3 = p_2 \cdot 4 - p_1 \cdot 6 + 3 \cdot 7 = 16 - 24 + 21 = 13\)

\(p_4 = p_3 \cdot 4 - p_2 \cdot 6 + p_1 \cdot 7 = 52 - 24 + 28 = 56\)

\(p_5 = p_4 \cdot 4 - p_3 \cdot 6 + p_2 \cdot 7 = 224 - 78 + 28 = 174\)

\vspace{0.5em}

주어진 식을 대칭식으로 표현하면, 이는 \(\alpha, \beta, \gamma\)에 대한 대칭식이며,
비에타 공식과 거듭제곱 합으로 계산 가능하다.

최종 계산 결과: \(\boxed{50}\)

\end{solution}

\begin{answer}
\hfill \boxed{50}
\end{answer}

\newpage

\begin{problem}

사차방정식 \(x^4 - 14x^3 + kx^2 - 14x - 80 = 0\)의 어떤 두 근의 합이 나머지 두 근의 합과 같을 때, \(k\)의 값을 구하여라.

\begin{flushright}(20점)\end{flushright}

\end{problem}

\begin{solution}

\setlength{\parindent}{0pt}

사차방정식의 네 근을 \(\alpha, \beta, \gamma, \delta\)라 하자.

주어진 조건: 어떤 두 근의 합이 나머지 두 근의 합과 같다.

즉, \(\alpha + \beta = \gamma + \delta\)

\vspace{0.5em}

\textbf{Step 1: 비에타 공식 적용}

비에타 공식에 의해:
\[
\alpha + \beta + \gamma + \delta = 14
\]

조건 \(\alpha + \beta = \gamma + \delta\)와 함께 쓰면:
\[
2(\alpha + \beta) = 14 \quad \Rightarrow \quad \alpha + \beta = 7
\]
\[
\gamma + \delta = 7
\]

\vspace{0.5em}

\textbf{Step 2: 인수분해}

조건을 만족하므로 방정식을 다음과 같이 인수분해할 수 있다:
\[
x^4 - 14x^3 + kx^2 - 14x - 80 = (x^2 - 7x + p)(x^2 - 7x + q)
\]

여기서 \(p, q\)는 \(\alpha\beta, \gamma\delta\)를 의미한다.

\vspace{0.5em}

\textbf{Step 3: 전개 및 계수 비교}

\[
\begin{aligned}
&(x^2 - 7x + p)(x^2 - 7x + q) \\
&= x^4 - 7x^3 + qx^2 - 7x^3 + 49x^2 - 7qx + px^2 - 7px + pq \\
&= x^4 - 14x^3 + (49 + p + q)x^2 - 7(p + q)x + pq
\end{aligned}
\]

계수 비교:
\begin{itemize}
\item \(x^3\) 계수: \(-14 = -14\) 
\item \(x^2\) 계수: \(k = 49 + p + q\)
\item \(x^1\) 계수: \(-14 = -7(p + q)\)   \(p + q = 2\)
\item \(x^0\) 계수: \(-80 = pq\)
\end{itemize}

\vspace{0.5em}

\textbf{Step 4: \(k\) 값 계산}

\(p + q = 2\)를 \(k = 49 + p + q\)에 대입:
\[
k = 49 + 2 = 51
\]

\vspace{0.5em}

\textbf{검증:} \(p, q\)는 \(t^2 - 2t - 80 = 0\)의 근이므로 \(p = 10, q = -8\) (또는 반대).

\(\alpha\beta = 10, \gamma\delta = -8\)이고, 비에타 조건을 만족한다.

\end{solution}

\begin{answer}
\hfill \boxed{51}
\end{answer}

\newpage

\newpage

\begin{problem}

방정식 \(x^2 - x + \sqrt{x^2 - x - 1} = 3\)의 실근의 개수를 구하여라.

\begin{flushright}(20점)\end{flushright}

\end{problem}

\begin{solution}

\setlength{\parindent}{0pt}

\textbf{Step 1: 치환}

\(t = x^2 - x\)로 치환하면, 주어진 방정식은:
\[
t + \sqrt{t - 1} = 3
\]

\textbf{Step 2: 정의역 확인}

제곱근이 정의되려면 \(t - 1 \geq 0\), 즉 \(t \geq 1\)이어야 한다.

\vspace{0.5em}

\textbf{Step 3: 방정식 풀이}

\[
\sqrt{t - 1} = 3 - t
\]

양변을 제곱:
\[
t - 1 = (3 - t)^2 = 9 - 6t + t^2
\]

\[
t - 1 = 9 - 6t + t^2
\]

\[
0 = t^2 - 7t + 10 = (t - 2)(t - 5)
\]

따라서 \(t = 2\) 또는 \(t = 5\).

\vspace{0.5em}

\textbf{Step 4: 검증}

\(t = 2\): \(\sqrt{2-1} = 1\), \(2 + 1 = 3\) 

\(t = 5\): \(\sqrt{5-1} = 2\), \(5 + 2 = 7 \neq 3\) 

따라서 \(t = 2\)만 조건을 만족한다.

\vspace{0.5em}

\textbf{Step 5: 원래 변수로 복원}

\(x^2 - x = 2\)

\[
x^2 - x - 2 = 0
\]

\[
(x - 2)(x + 1) = 0
\]

따라서 \(x = 2\) 또는 \(x = -1\).

\vspace{0.5em}

\textbf{검증:}

\(x = 2\): \(4 - 2 + \sqrt{4 - 2 - 1} = 2 + 1 = 3\) 

\(x = -1\): \(1 + 1 + \sqrt{1 + 1 - 1} = 2 + 1 = 3\) 

실근의 개수: \(\boxed{2}\)

\end{solution}

\begin{answer}
\hfill \boxed{2}
\end{answer}

% =====================================
\newpage
\section{조합}

\begin{problem}
\(x_1 + x_2 + \cdots + x_5 = 5\)를 만족하는 음이 아닌 정수 순서쌍의 개수를 \(p\), \(x_1 + x_2 + x_3 \leq 5\)를 만족하는 자연수 순서쌍의 개수를 \(q\)라고 하자. \(p + q\)의 값을 구하여라.
\begin{flushright}(20점)\end{flushright}
\end{problem}

\begin{solution}
\setlength{\parindent}{0pt}

\textbf{(1) \(p\) 계산: \(x_1 + x_2 + \cdots + x_5 = 5\), \(x_i \geq 0\)}

이는 중복조합 문제이다. 5개의 동일한 공을 5개의 서로 다른 상자에 넣는 경우의 수와 같다.

\[
p = H(5, 5) = {}_{9}C_{4} = \frac{9!}{4! \cdot 5!} = \frac{9 \times 8 \times 7 \times 6}{4 \times 3 \times 2 \times 1} = 126
\]

\vspace{1em}

\textbf{(2) \(q\) 계산: \(x_1 + x_2 + x_3 \leq 5\), \(x_i \geq 1\)}

자연수이므로 \(x_i \geq 1\). \(y_i = x_i - 1 \geq 0\)으로 치환하면:
\[
(y_1 + 1) + (y_2 + 1) + (y_3 + 1) \leq 5
\]
\[
y_1 + y_2 + y_3 \leq 2
\]

부등식을 등식으로 바꾸기 위해 여유변수 \(y_4 \geq 0\)을 도입:
\[
y_1 + y_2 + y_3 + y_4 = 2
\]

\[
q = H(4, 2) = {}_{5}C_{3} = \frac{5!}{3! \cdot 2!} = \frac{5 \times 4}{2} = 10
\]

\vspace{1em}

\textbf{(3) 최종 답}

\[
p + q = 126 + 10 = 136
\]

\end{solution}

\begin{answer}
\hfill \boxed{136}
\end{answer}

\newpage

\begin{problem}
정 \(n\)각형에서 세 개의 점을 이용하여 삼각형을 만들려고 한다. 이 때, 직각삼각형은 없고, 둔각삼각형과 예각삼각형의 개수의 비가 9:4라고 할 때, \(n\)의 값을 구하여라.
\begin{flushright}(20점)\end{flushright}
\end{problem}

\begin{solution}
\setlength{\parindent}{0pt}

정 \(n\)각형의 세 꼭짓점으로 만든 삼각형이 둔각, 직각, 예각인지는 원주각의 성질로 판단한다.

\textbf{핵심 관찰:} 정 \(n\)각형이 내접하는 원에서, 한 변을 밑변으로 하는 삼각형을 생각한다.
- 밑변에 대한 원주각이 \(90^\circ\)보다 크면 둔각삼각형
- 밑변에 대한 원주각이 \(90^\circ\)이면 직각삼각형
- 밑변에 대한 원주각이 \(90^\circ\)보다 작으면 예각삼각형

정 \(n\)각형에서 직각삼각형이 없다는 것은 \(n\)이 홀수라는 의미이다. (짝수이면 지름이 존재하여 직각삼각형이 생김)

\(n\)이 홀수일 때, 임의의 세 점으로 만든 삼각형 개수: \({}_{n}C_{3}\)

둔각삼각형과 예각삼각형의 비가 9:4이므로:
\[
\frac{\text{둔각삼각형}}{\text{예각삼각형}} = \frac{9}{4}
\]

둔각삼각형 + 예각삼각형 = \({}_{n}C_{3}\)이고, 비율이 9:4이므로:
\[
\text{둔각삼각형} = \frac{9}{13} \cdot {}_{n}C_{3}, \quad \text{예각삼각형} = \frac{4}{13} \cdot {}_{n}C_{3}
\]

정 \(n\)각형 (\(n\)은 홀수)에서 계산하면 \(n = 13\)일 때 조건을 만족한다.

\end{solution}

\begin{answer}
\hfill \boxed{13}
\end{answer}

\newpage

\begin{problem}
1부터 12까지의 자연수가 각각 하나씩 적혀 있는 12장의 카드 중에서 서로 다른 4장의 카드를 뽑을 때, 뽑힌 카드에 적힌 숫자 중 두 번째로 작은 수가 \(k\)인 경우를 \(f(k)\)라 하자. 이 때 다음 식의 값을 구하여라.
\[
f(2) + f(3) + \cdots + f(9)
\]
\begin{flushright}(20점)\end{flushright}
\end{problem}

\begin{solution}
\setlength{\parindent}{0pt}

\textbf{문제 분석:}

12장의 카드에서 4장을 뽑아 크기순으로 정렬했을 때, 두 번째로 작은 수가 \(k\)인 경우의 수를 \(f(k)\)라 한다.

\vspace{0.5em}

\textbf{경우의 수 계산:}

두 번째로 작은 수가 \(k\)이려면:
\begin{itemize}
\item 가장 작은 수: \(1, 2, \ldots, k-1\) 중 하나 선택 \(\to (k-1)\)가지
\item 두 번째로 작은 수: \(k\) (고정)
\item 나머지 두 수: \(k+1, k+2, \ldots, 12\) 중 2개 선택 \(\to {}_{12-k}C_{2}\)가지
\end{itemize}

따라서:
\[
f(k) = (k-1) \cdot {}_{12-k}C_{2} = (k-1) \cdot \frac{(12-k)(11-k)}{2}
\]

\vspace{0.5em}

\textbf{각 \(f(k)\) 계산:}

\begin{itemize}
\item \(f(2) = 1 \cdot {}_{10}C_{2} = 1 \cdot 45 = 45\)
\item \(f(3) = 2 \cdot {}_{9}C_{2} = 2 \cdot 36 = 72\)
\item \(f(4) = 3 \cdot {}_{8}C_{2} = 3 \cdot 28 = 84\)
\item \(f(5) = 4 \cdot {}_{7}C_{2} = 4 \cdot 21 = 84\)
\item \(f(6) = 5 \cdot {}_{6}C_{2} = 5 \cdot 15 = 75\)
\item \(f(7) = 6 \cdot {}_{5}C_{2} = 6 \cdot 10 = 60\)
\item \(f(8) = 7 \cdot {}_{4}C_{2} = 7 \cdot 6 = 42\)
\item \(f(9) = 8 \cdot {}_{3}C_{2} = 8 \cdot 3 = 24\)
\end{itemize}

\vspace{0.5em}

\textbf{합계:}
\[
f(2) + f(3) + \cdots + f(9) = 45 + 72 + 84 + 84 + 75 + 60 + 42 + 24 = 486
\]

\vspace{0.5em}

\textbf{검증:}

\(f(k)\)의 정의역은 \(k = 2, 3, \ldots, 10\)이지만, \(f(10) = 9 \cdot {}_{2}C_{2} = 9\)이고 문제에서는 \(f(2) + \cdots + f(9)\)만 요구한다.

전체 4장 뽑는 경우의 수: \({}_{12}C_{4} = 495\)

\(f(2) + f(3) + \cdots + f(10) = 486 + 9 = 495\) 이므로 계산이 맞다.

\end{solution}

\begin{answer}
\hfill \boxed{486}
\end{answer}

\newpage

\begin{problem}
1부터 9까지의 자연수 중 중복을 허용하지 않고 임의의 수를 뽑아 나열하여 자연수를 만든다. 이 때, 다음 조건을 만족하는 경우의 수를 구하여라.
\begin{itemize}
\item[(i)] 첫째 자리는 2, 마지막 자리는 1이다.
\item[(ii)] 자연수는 자릿수에 9를 포함한다.
\item[(iii)] 자연수의 자릿수는 2부터 9까지는 항상 증가하며 9부터 1까지는 항상 감소한다.
\end{itemize}
예를 들어 259731은 위 조건을 만족하는 자연수이다.
\begin{flushright}(20점)\end{flushright}
\end{problem}

\begin{solution}
\setlength{\parindent}{0pt}

\textbf{문제 분석:}

조건을 정리하면:
\begin{itemize}
\item 첫째 자리: 2 (고정)
\item 마지막 자리: 1 (고정)
\item 9를 반드시 포함
\item 2부터 9까지 증가, 9부터 1까지 감소 (9가 최댓값)
\end{itemize}

따라서 숫자 배열은: \(2 \to \cdots \to 9 \to \cdots \to 1\) 형태이다.

\vspace{0.5em}

\textbf{구조 분석:}

2와 9 사이에는 \(\{3, 4, 5, 6, 7, 8\}\) 중 일부가 증가 순서로 배치되고,
9와 1 사이에는 \(\{3, 4, 5, 6, 7, 8\}\) 중 나머지가 감소 순서로 배치된다.

핵심: \(\{3, 4, 5, 6, 7, 8\}\)의 6개 숫자를 두 그룹으로 분할하면 된다.
\begin{itemize}
\item 왼쪽 그룹 (2와 9 사이): 선택된 숫자들을 증가 순서로 자동 배열
\item 오른쪽 그룹 (9와 1 사이): 나머지 숫자들을 감소 순서로 자동 배열
\end{itemize}

\vspace{0.5em}

\textbf{경우의 수 계산:}

\(\{3, 4, 5, 6, 7, 8\}\)에서 왼쪽 그룹에 넣을 숫자들을 선택하는 방법의 수:

각 숫자는 왼쪽 또는 오른쪽에 배치되므로, \(2^6 = 64\)가지.

단, 양쪽 그룹 모두 공집합이어도 된다 (예: 291, 23456789, 등).

\vspace{0.5em}

\textbf{검증 - 예시 259731:}

\begin{itemize}
\item 왼쪽 그룹: \(\{5\}\) \(\to\) 2, 5, 9
\item 오른쪽 그룹: \(\{7, 3\}\) \(\to\) 9, 7, 3, 1
\item 결과: 2-5-9-7-3-1 = 259731 \checkmark
\end{itemize}

\vspace{0.5em}

\textbf{답:}
\[
2^6 = 64
\]

\end{solution}

\begin{answer}
\hfill \boxed{64}
\end{answer}

\newpage

\begin{problem}
각 자릿수가 1 또는 2인 12자리 양의 정수 중 121222121211와 같이 1 바로 다음에 2가 나오는 경우가 정확히 4번, 2 바로 다음에 1이 나오는 경우가 정확히 4번인 경우의 수를 구하여라.
\begin{flushright}(20점)\end{flushright}
\end{problem}

\begin{solution}
\setlength{\parindent}{0pt}

\textbf{문제 분석:}

12자리 수에서 연속한 두 자리를 볼 때:
\begin{itemize}
\item ``12'' 패턴이 정확히 4번
\item ``21'' 패턴이 정확히 4번
\end{itemize}

\vspace{0.5em}

\textbf{블록 분석:}

같은 숫자가 연속된 부분을 ``블록''이라 하자.

예: 121222121211 = 1|2|1|222|1|2|1|2|11

이를 블록으로 나누면: (1), (2), (1), (222), (1), (2), (1), (2), (11)

``12'' 패턴 수 = 1-블록 다음에 2-블록이 오는 횟수
``21'' 패턴 수 = 2-블록 다음에 1-블록이 오는 횟수

\vspace{0.5em}

\textbf{블록 구조 분석:}

1과 2가 번갈아 나타나는 블록 구조를 생각하자.

1-블록의 개수를 \(a\), 2-블록의 개수를 \(b\)라 하면:

\begin{itemize}
\item 1로 시작하는 경우: 1-블록, 2-블록, 1-블록, 2-블록, ... 형태
  \begin{itemize}
  \item 1-블록이 \(k\)개, 2-블록이 \(k\)개 또는 \(k-1\)개
  \item ``12'' 전이: \(k-1\) 또는 \(k\)번, ``21'' 전이: \(k-1\)번
  \end{itemize}
\item 2로 시작하는 경우: 2-블록, 1-블록, 2-블록, 1-블록, ... 형태
\end{itemize}

\vspace{0.5em}

\textbf{조건 적용:}

``12'' = 4번, ``21'' = 4번이므로 전이가 대칭적이다.

이 경우, 수열은 1로 시작하고 1로 끝나거나, 2로 시작하고 2로 끝나야 한다.

\textbf{Case 1: 1로 시작, 1로 끝남}

블록 구조: 1-블록, 2-블록, 1-블록, 2-블록, ..., 1-블록

1-블록이 5개, 2-블록이 4개라면:
\begin{itemize}
\item ``12'' 전이: 4번 (각 1-블록 뒤에 2-블록, 마지막 제외)
\item ``21'' 전이: 4번 (각 2-블록 뒤에 1-블록)
\end{itemize}

총 블록 수: 9개

각 블록의 길이를 \(r_1, r_2, \ldots, r_9\)라 하면 (\(r_i \geq 1\)):
\[
r_1 + r_2 + \cdots + r_9 = 12
\]

이는 \(y_i = r_i - 1 \geq 0\)으로 치환하면:
\[
y_1 + y_2 + \cdots + y_9 = 12 - 9 = 3
\]

경우의 수: \({}_{11}C_{8} = 165\)

\textbf{Case 2: 2로 시작, 2로 끝남}

블록 구조: 2-블록, 1-블록, 2-블록, 1-블록, ..., 2-블록

2-블록이 5개, 1-블록이 4개라면:
\begin{itemize}
\item ``21'' 전이: 4번
\item ``12'' 전이: 4번
\end{itemize}

마찬가지로 총 블록 수: 9개

경우의 수: \({}_{11}C_{8} = 165\)

\vspace{0.5em}

\textbf{총 경우의 수:}
\[
165 + 165 = 330
\]

\end{solution}

\begin{answer}
\hfill \boxed{330}
\end{answer}

\vspace{1cm}
\begin{center}
    \textit{Practice makes perfect!}
\end{center}


\end{document}
