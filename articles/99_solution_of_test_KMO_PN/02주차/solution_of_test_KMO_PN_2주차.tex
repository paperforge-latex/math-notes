\documentclass[12pt,a4paper]{article}
\usepackage{kotex}            % 한글 지원
\usepackage{amsmath,amssymb}  % 수학 기호 및 수식 패키지
\usepackage{amsthm}           % 정리, 증명 환경
\usepackage{graphicx}         % 이미지 삽입
\usepackage{geometry}         % 페이지 여백 설정
\usepackage{hyperref}         % 하이퍼링크
\usepackage{color}            % 색상 지원
\usepackage{etoolbox}         % 환경 후킹을 위한 패키지
\usepackage{tikz}
\usetikzlibrary{fit,calc}

% 페이지 여백 설정
\geometry{margin=2.5cm}

% 정리, 정의, 예제 환경 설정
% \theoremstyle{definition}

\newtheoremstyle{test_form}% name
  {10pt}% Space above
  {10pt}% Space below
  {\normalfont\setlength{\parindent}{20pt}} % Body font(+ 본문 들여쓰기)
  {0em}% Indent amount
  {\bfseries}% Theorem head font
  {}% Punctuation after theorem head
  {\newline}% Space after theorem head (line break!)
  {}% Theorem head spec
\theoremstyle{test_form}
\newtheorem{problem}{문제}[section]
\newtheorem*{solution}{풀이}
\newtheorem*{answer}{정답}

% 섹션마다 문제 번호 리셋
\makeatletter
\@addtoreset{problem}{section}
\makeatother

% 문제 번호를 섹션 번호 없이 표시
\renewcommand{\theproblem}{\arabic{problem}}

% --- 자동 목차 등록 ---
\newcommand{\tocaddsolution}{%
  \phantomsection
  \addcontentsline{toc}{subsubsection}{풀이}%
}
\newcommand{\tocaddanswer}{%
  \phantomsection
  \addcontentsline{toc}{subsubsection}{정답}%
}

% problem 환경의 헤더 부분에 목차 추가
\makeatletter
\let\old@problem\problem
\renewcommand{\problem}{%
  \old@problem
  \phantomsection
  \addcontentsline{toc}{subsection}{문제 \theproblem}%
}
\makeatother

\AtBeginEnvironment{solution}{\tocaddsolution}
\AtBeginEnvironment{answer}{\tocaddanswer}

\title{KMO 대비반 중급 2주차 테스트}
\author{Taeyang Lee}
\date{\today}

\begin{document}

\maketitle
\tableofcontents  % 목차 자동 생성

\newpage

% =====================================
\section{대수}

\begin{problem}
    
    $f(x) = \dfrac{4^x}{4^x+2}$ 이라 하자. 
    합 
    $f\!\left(\dfrac{1}{801}\right) + f\!\left(\dfrac{2}{801}\right) + f\!\left(\dfrac{3}{801}\right) + \cdots + f\!\left(\dfrac{800}{801}\right)$
    의 값을 구하여라.
    \begin{flushright}(20점)\end{flushright}
    
\end{problem}

\begin{solution}
\[
f(x)=\frac{4^x}{4^x+2}.
\]
대칭성을 이용하면
\[
f(x)+f(1-x)=\frac{4^x}{4^x+2}+\frac{4^{1-x}}{4^{1-x}+2}
=\frac{4^x}{4^x+2}+\frac{2}{4^x+2}=1.
\]
따라서
\[
\sum_{k=1}^{800} f\!\left(\frac{k}{801}\right)
=\sum_{k=1}^{400}\left\{f\!\left(\frac{k}{801}\right)
+f\!\left(1-\frac{k}{801}\right)\right\}
=400.
\]
\end{solution}

\begin{answer}
\hfill \boxed{400}
\end{answer}

\newpage

\begin{problem}
다음을 정수 범위에서 최대한 인수분해하였을 때, 곱해져 있는 항의 개수를 구하여라.
\[
a^{3}(c^{2}-b^{2})+b^{3}(a^{2}-c^{2})+c^{3}(b^{2}-a^{2}).
\]
(예: \(ab(ab+bc)\)는 3개의 항이 곱해져 있는 식이다.)
\begin{flushright}(20점)\end{flushright}
\end{problem}

\begin{solution}

\setlength{\parindent}{0pt}

\textbf{[풀이 1: 교대식의 성질 이용]}

주어진 식은 교대식이다. 교대식의 성질에 의해 다음과 같이 표현할 수 있다:
\[
\text{(교대식)} = \text{(교대식)} \times \text{(대칭식)}.
\]

인수정리에 의해 \((a-b)(b-c)(c-a)\)인 교대식이 인수로 존재하므로, 나머지 부분은 2차 대칭식이어야 한다.

2차 대칭식은 기본대칭식
\[
e_1=a+b+c,\quad e_2=ab+bc+ca,\quad e_3=abc
\]
의 사칙연산으로 표현되므로, \(Ae_1^2 + Be_1e_2 + Ce_3\) 형태로 나타낼 수 있다.

계수를 비교하면 \(A=1\), \(B=1\), \(C=0\)이므로,
\[
a^3(c^2-b^2)+b^3(a^2-c^2)+c^3(b^2-a^2)
=(a-b)(b-c)(c-a)(a^2+b^2+c^2+ab+bc+ca).
\]

\textbf{[풀이 2: 직접 인수분해]}

주어진 식을 정리하면,
\[
\begin{aligned}
&a^3(c^2-b^2)+b^3(a^2-c^2)+c^3(b^2-a^2)\\
&=a^3c^2-a^3b^2+b^3a^2-b^3c^2+c^3b^2-c^3a^2\\
&=(a^3c^2-b^3c^2)+(b^3a^2-c^3a^2)+(c^3b^2-a^3b^2)\\
&=c^2(a^3-b^3)+a^2(b^3-c^3)+b^2(c^3-a^3)\\
&=c^2(a-b)(a^2+ab+b^2)+a^2(b-c)(b^2+bc+c^2)+b^2(c-a)(c^2+ca+a^2)\\
&=(a-b)(b-c)(c-a)(a^2+b^2+c^2+ab+bc+ca).
\end{aligned}
\]

따라서 곱해진 항의 개수는 \textbf{4개}이다.
\end{solution}

\begin{answer}
\hfill \boxed{4}
\end{answer}

\newpage

\begin{problem}
다음 연립방정식의 해를 \(x,y,z\)라 할 때, \(x^{4}+y^{4}+z^{4}\)의 값을 구하여라.
\[
\begin{cases}
x+y+z=4,\\[2pt]
x^{2}+y^{2}+z^{2}=14,\\[2pt]
x^{3}+y^{3}+z^{3}=34.
\end{cases}
\]
\begin{flushright}(20점)\end{flushright}
\end{problem}

\begin{solution}
    \setlength{\parindent}{0pt}
    
    세 문자 $x,y,z$에 대해 주어진 식은 \emph{대칭식}(Symmetric Polynomial)이다.  
    \\따라서 기본대칭식
    \[
    e_1 = x+y+z, \quad e_2 = xy+yz+zx, \quad e_3 = xyz
    \]
    의 값만 알면 모든 대칭식의 값을 구할 수 있다.
    \\
    문제에서 제시된 정보는 멱합(Power Sum)
    \[
    p_1 = x+y+z, \quad p_2 = x^2+y^2+z^2, \quad p_3 = x^3+y^3+z^3
    \]
    이다. 
    \\이들은 $e_1,e_2,e_3$와 \textbf{뉴턴 항등식}(Newton's Identities) 관계를 가진다:
    \begin{align}
        p_1 &= e_1 \nonumber \\
        p_2 &= e_1p_1 - 2e_2 \nonumber \\
        p_3 &= e_1p_2 - e_2p_1 + 3e_3 \nonumber
    \end{align}
    
    문제에서 $p_1=4,\;p_2=14,\;p_3=34$가 주어졌다. \\
    이를 대입하면
    \begin{align}
        e_1 &= 4 \nonumber \\
        14 &= 4\cdot4 - 2e_2 \nonumber \\
        34 &= 4\cdot14 - 1\cdot4 + 3e_3 \nonumber \\
        \therefore e_2 &= 1, \quad e_3 = -6 \nonumber
    \end{align}
    
    이제 $p_4 = x^4+y^4+z^4$를 계산한다:
    \[
    p_4 = e_1p_3 - e_2p_2 + e_3p_1
    = 4\cdot34 - 1\cdot14 + (-6)\cdot4.
    \]
    \[
    =136 - 14 - 24 = 98.
    \]    
\end{solution}
    

\begin{answer}
\hfill \boxed{98}
\end{answer}

\newpage

\begin{problem}
\(\sqrt[4]{x}+\sqrt[4]{97-x}=5\)일 때, \(\sqrt[4]{x}\cdot\sqrt[4]{97-x}\)의 값을 구하여라.
\begin{flushright}(20점)\end{flushright}
\end{problem}

\begin{solution}
    \setlength{\parindent}{0pt}
    주어진 식에서 다음과 같이 치환하자 \\
    $a=\sqrt[4]{x},\; b=\sqrt[4]{97-x}$
    \\\\
    문제의 조건은
    \[
    e_1 = a+b = 5
    \]
    이며, 구하는 값은 $e_2 = ab$이다.
    \\
    또한 $a^4+b^4=97$이 주어졌다.
    \[
    a^2+b^2 = e_1^2 - 2e_2 = 25 - 2e_2
    \]
    이고, 따라서
    \[
    a^4+b^4 = (a^2+b^2)^2 - 2(ab)^2 = (25-2e_2)^2 - 2e_2^2.
    \]
    
    주어진 조건 $a^4+b^4=97$을 대입하면
    \[
    (25-2e_2)^2 - 2e_2^2 = 97.
    \]
    
    이를 전개하면
    \[
    625 - 100e_2 + 4e_2^2 - 2e_2^2 = 97,
    \]
    \[
    2e_2^2 - 100e_2 + 528 = 0,
    \]
    \[
    e_2^2 - 50e_2 + 264 = 0.
    \]
    
    따라서
    \[
    e_2 = 44 \quad \text{또는} \quad e_2 = 6.
    \]
    
    그런데 $ab=e_2 \leq \dfrac{(a+b)^2}{4} = \dfrac{25}{4}=6.25$가 되어야 하므로 가능한 해는
    \[
    e_2 = 6.
    \]
    
\end{solution}
        
\begin{answer}
\hfill \boxed{6}
\end{answer}

\newpage

\begin{problem}
\(m^{3}+n^{3}+99mn=33^{3}\)을 만족시키는 양의 정수 \((m,n)\)의 순서쌍의 개수를 구하여라.
\begin{flushright}(20점)\end{flushright}
\end{problem}

   
\begin{solution}
\setlength{\parindent}{0pt}

주어진 식을 정리하면
\[
m^3+n^3+3mn(m+n)=(m+n)^3
\]
조건은 \(3mn(m+n)=99mn\)이므로 \(m+n=33\) \\
\\양의 정수 해는 \((m,n)=(1,32),(2,31),\ldots,(32,1)\)로 32개이다.
\end{solution}

\begin{answer}
\hfill \boxed{32}
\end{answer}

% =====================================
\newpage
\section{조합}

\begin{problem}
어느 부부가 아홉 쌍의 부부를 집으로 초대하여 파티를 열었다. 이 자리에 모인 열 쌍의 부부는 서로 아는 사이도 있고, 처음 만나는 사이도 있다. 이들 가운데 서로 알던 사람들은 악수를 하지 않았지만, 처음 만나는 사람들은 정중하게 악수를 한 번씩 나누었다. 저녁 식사가 끝나고 집 주인은 그 자리에 모인 19명(집 주인의 부인과 손님들)에게 오늘 모임에서 악수를 몇 번 하였는지 질문하였다. 놀랍게도 이들이 악수한 횟수는 모두 달랐다. 이때 집 주인의 부인은 악수를 몇 번 하였는지 구하시오.
\begin{flushright}(20점)\end{flushright}
\end{problem}


\begin{solution}
    \setlength{\parindent}{0pt}

    19명의 악수 횟수는 자기 자신과 와이프를 제외하므로 \\
    최소값 0회, 최대값 18회이고, \(0,1,2,\dots,18\) 으로 유일하게 결정된다.\\
    부부는 서로 악수를 하지 않으므로 한 쌍의 악수 횟수 합은 항상 \(18\)이 된다.  \\

    18회 악수한 사람은 본인의 배우자를 제외한 모든 사람과 악수했다.  \\
    따라서 0회 악수한 사람은 그 사람의 배우자이다.  \\
    마찬가지로 17회와 1회, 16회와 2회, \ldots가 각각 부부이다. \\

    이러한 쌍은 (18, 0), (17, 1), (16, 2), \ldots, (10, 8), (9, 9)로 10쌍이다. \\
    집 주인 부부가 (9, 9)에 해당하므로, 집 주인의 부인은 9회 악수했다.


    \begin{center}
    \begin{tikzpicture}[scale=1.0]
      \def\n{19}
      \foreach \k in {0,...,18}{
        \node[circle,draw,minimum size=6mm,inner sep=0pt] (v\k) at (90-\k*360/\n:3.0) {\small \k};
      }
      % 짝(pair) 표시: k 와 18-k 연결
      \draw[gray, dashed] (v0) -- (v18);
      \draw[gray, dashed] (v1) -- (v17);
      \draw[gray, dashed] (v2) -- (v16);
      \draw[gray, dashed] (v3) -- (v15);
      \draw[gray, dashed] (v4) -- (v14);
      \draw[gray, dashed] (v5) -- (v13);
      \draw[gray, dashed] (v6) -- (v12);
      \draw[gray, dashed] (v7) -- (v11);
      \draw[gray, dashed] (v8) -- (v10);
      % (9,9) 강조
      \node[draw=red,thick,circle,inner sep=3pt] at (v9) {};
      \node at (0,-3.9) {\small 악수 횟수 페어: 합이 항상 18. (9,9)가 집주인 부부};
    \end{tikzpicture}
    \end{center}
    \end{solution}
    

\begin{answer}
\hfill \boxed{9}
\end{answer}

\newpage

\begin{problem}
정팔각형의 변을 따라 움직이는 로봇이 있다. 이 로봇은 정팔각형의 한 변을 지나가는데 1분이 걸리며, 각 꼭짓점에서 가던 방향으로 계속 가거나 반대 방향으로 방향을 바꿀 수 있다. 이 로봇이 한 꼭짓점 \(A\)에서 출발하여 10분 동안 계속 움직여 다시 꼭짓점 \(A\)로 돌아오는 경우의 수를 구하여라.
\begin{flushright}(20점)\end{flushright}
\end{problem}

\begin{solution}
    \setlength{\parindent}{0pt}
    정팔각형에서 한 방향으로 가면 +1, 반대 방향으로 가면 -1이라 하자. 10분 후 출발점으로 돌아오려면 총 이동 거리가 8의 배수여야 한다.

10번 이동 중 \(m\)번 정방향, \(10-m\)번 역방향으로 가면 총 이동은 \(m-(10-m)=2m-10\)이다. 이 값이 \(\pm8\) 또는 0이어야 한다. \\

\begin{itemize}
\item \(2m-10=8 \Rightarrow m=9\): \(_{10}C_9=10\)
\item \(2m-10=0 \Rightarrow m=5\): \(_{10}C_5=252\)
\item \(2m-10=-8 \Rightarrow m=1\): \(_{10}C_1=10\)
\end{itemize}

따라서 총 경우의 수는 \(10+252+10=272\).
\end{solution}

\begin{answer}
\hfill \boxed{272}
\end{answer}

\newpage

\begin{problem}
여섯 쌍의 부부가 모임에 참가하였다. 사회자가 그 중 일곱 명을 뽑아 선물을 준다. 이때, 선물을 하나도 받지 못한 부부가 오직 한 쌍일 경우의 수를 구하여라.
\begin{flushright}(20점)\end{flushright}
\end{problem}

\begin{solution}
    \setlength{\parindent}{0pt}

    선물을 받지 못할 부부 한 쌍을 선택: \( _6 C_1 = 6\).

    나머지 5쌍(10명) 중 7명을 뽑되, 각 쌍에서 최소 한 명은 선택되어야 한다.

    5쌍 중 3쌍에서 한 명씩만, 2쌍에서 두 명씩 선택한다.

\begin{itemize}
\item 한 명만 선택할 3쌍을 고름: \(_5C_3=10\)
\item 각 쌍에서 한 명을 고름: \(2^3=8\)
\end{itemize}

따라서 총 경우의 수는 \(6 \times 10 \times 8 = 480\).
\end{solution}

\begin{answer}
\hfill \boxed{480}
\end{answer}

\newpage

\begin{problem}
\(1,2,2,3,3,5,7,9\)의 숫자가 각각 하나씩 적힌 8장의 카드 중에서 2장 이상의 카드를 동시에 뽑을 때, 카드에 적힌 숫자를 모두 곱한 값이 될 수 있는 서로 다른 수의 개수를 구하여라.
\begin{flushright}(20점)\end{flushright}
\end{problem}

\begin{solution}
카드: \(1, 2, 2, 3, 3, 5, 7, 9\). 소인수분해하면 \(9=3^2\)이므로 가능한 곱은 \(2^a \cdot 3^b \cdot 5^c \cdot 7^d\) 형태이다.

\begin{itemize}
\item \(a \in \{0, 1, 2\}\): 3가지
\item \(b \in \{0, 1, 2, 3, 4\}\): 5가지 (3이 2개, 9=\(3^2\)가 1개)
\item \(c \in \{0, 1\}\): 2가지
\item \(d \in \{0, 1\}\): 2가지
\end{itemize}

총 \(3 \times 5 \times 2 \times 2 = 60\)가지. 단, 1만 선택하는 경우(카드 1장)는 제외되므로 \(60 - 1 = 59\).
\end{solution}

\begin{answer}
\hfill \boxed{59}
\end{answer}

\newpage

\begin{problem}
정사면체 \(ABCD\)의 6개 모서리 중에서 몇 개의 모서리를 골라 색을 칠한다. 이때 색을 칠한 모서리들을 따라 4개의 꼭짓점이 모두 연결되는 경우의 수를 구하여라.
\begin{flushright}(20점)\end{flushright}
\end{problem}


\begin{solution}

\textbf{[풀이 1: 모서리 개수별 분류]}

\setlength{\parindent}{0pt}

정사면체는 4개의 꼭짓점과 6개의 모서리를 가진다. 4개의 꼭짓점이 모두 연결되려면 색칠한 모서리를 따라 어떤 두 꼭짓점 사이에도 경로가 존재해야 한다.

모서리 \(k\)개를 칠하는 경우를 분류하면:

\textbf{(i) \(k=6\)}: 모든 모서리를 칠함.
\[
c_6=1
\]

\textbf{(ii) \(k=5\)}: 5개를 칠함. 어떤 모서리 하나를 빼도 나머지로 4개 꼭짓점을 연결할 수 있다.
\[
c_5=_6C_1=6
\]

\textbf{(iii) \(k=4\)}: 4개를 칠함. 정사면체에서 4개 모서리로는 고립된 꼭짓점이 생기지 않는다. 즉 \(_6C_4\) 모두 연결된다.
\[
c_4=_6C_4=15
\]

\textbf{(iv) \(k=3\)}: 3개를 칠함. 두 경우로 나뉜다.
\begin{itemize}
\item 한 면의 3개 모서리를 모두 선택(정삼각형): 한 꼭짓점이 고립되므로 \emph{연결되지 않음}. 이런 경우는 4가지.
\item 나머지 경우: 4개 꼭짓점을 모두 거쳐가는 트리 형태이므로 \emph{연결됨}. 전체 \(_6C_3=20\)에서 4가지를 제외.
\end{itemize}
\[
c_3=20-4=16
\]

\textbf{(v) \(k\le 2\)}: 4개 꼭짓점을 모두 연결하려면 최소 3개 모서리가 필요하므로
\[
c_2=c_1=c_0=0
\]

따라서 전체 연결 경우의 수는
\[
c_3+c_4+c_5+c_6=16+15+6+1=38.
\]

\vspace{0.5cm}

\textbf{[풀이 2: 여사건 이용 (포함-배제 원리)]}

정사면체는 4개 꼭짓점과 6개 모서리를 가진다. 각 모서리를 칠하거나 칠하지 않는 경우는 총 \(2^6 = 64\)가지다.

이 중에서 4개 꼭짓점이 \emph{연결되지 않는} 경우를 구하여 전체에서 빼면 된다.

\textbf{연결되지 않는 경우:}
\begin{itemize}
\item \textbf{Type 1}: 고립된 꼭짓점이 존재하는 경우
  \begin{itemize}
  \item 1개 꼭짓점 고립: 그 꼭짓점과 연결된 3개 모서리를 모두 칠하지 않음 → \(_4C_1 \times 2^3 = 32\)
  \item 2개 꼭짓점 고립: \(_4C_2 \times 2^0 = 6\)
  \item 3개 이상 고립: \(_4C_3 + _4C_4 = 5\)
  \end{itemize}

\item \textbf{Type 2}: 고립된 점은 없지만 두 그룹으로 분리되는 경우
  \begin{itemize}
  \item 2-2 분할: 정사면체를 두 쌍으로 나누는 방법은 3가지
  \item 각 분할에서 그룹 내부 연결만 있는 경우를 계산
  \end{itemize}
\end{itemize}

포함-배제 원리를 적용하고 중복을 제거하면, 연결되지 않는 경우는 총 26가지이다.

따라서 연결되는 경우: \(64 - 26 = 38\).

\vspace{0.5cm}

\textbf{[풀이 3: 그래프 이론 관점]}

\textbf{※ 그래프 이론 기본 개념:}

\textbf{(1) 완전그래프(Complete Graph) \(K_n\):}

\(n\)개의 꼭짓점을 가지며, 임의의 서로 다른 두 꼭짓점이 모두 \textbf{간선}(edge, 모서리)으로 연결되어 있는 그래프.

여기서 \textbf{간선}이란 두 꼭짓점을 잇는 선을 의미하며, 영어로는 edge, 즉 모서리를 뜻한다. 정사면체의 모서리가 바로 간선의 예시이다.

\(K_n\)의 간선 개수는 \(_nC_2 = \frac{n(n-1)}{2}\)개이다.

예를 들어, \(K_3\)는 정삼각형 (꼭짓점 3개, 간선 3개), \(K_4\)는 정사면체 (꼭짓점 4개, 간선 6개)이다.

\textbf{(2) 차수(Degree):}

한 꼭짓점의 \textbf{차수}란 그 꼭짓점에 연결된 간선의 개수를 의미한다.

예시:
\begin{itemize}
\item 완전그래프 \(K_4\)에서 각 꼭짓점의 차수는 3 (나머지 3개 꼭짓점과 모두 연결)
\item 한 꼭짓점이 \textbf{고립}되었다 = 그 꼭짓점의 차수가 0
\item 연결된 그래프 = 모든 꼭짓점의 차수 \(\geq 1\)
\end{itemize}

\textbf{(3) 트리(Tree):}

\(n\)개의 꼭짓점을 가진 \textbf{연결 그래프} 중에서 사이클(cycle)이 없는 그래프. 트리는 정확히 \(n-1\)개의 간선을 가진다.

예: 4개 꼭짓점을 연결하는 트리는 정확히 3개 간선 필요.

\vspace{0.3cm}

정사면체는 4개 꼭짓점과 6개 모서리를 가지므로 완전그래프 \(K_4\)와 대응된다.
\[
K_4\text{의 간선 개수} = _4C_2 = \frac{4 \times 3}{2} = 6
\]

이 문제는 \(K_4\)의 6개 간선 중 일부를 선택하여 4개 꼭짓점이 모두 \textbf{연결된} 부분그래프를 만드는 경우의 수를 구하는 것이다.

\textbf{연결 그래프의 조건:}
\begin{itemize}
\item \textbf{필요조건 1}: 간선 개수 \(\geq n-1\) (4개 꼭짓점을 연결하려면 최소 3개 간선 필요)
\item \textbf{필요조건 2}: 각 꼭짓점의 차수 \(\geq 1\) (모든 꼭짓점이 최소 하나의 간선과 연결)
\end{itemize}

\textbf{케이스별 계산:}

\textbf{(i)} \(k=6\): 완전그래프 \(K_4\) 자체. 당연히 연결.
\[
_6C_6 = 1
\]

\textbf{(ii)} \(k=5\): \(K_4\)에서 간선 하나를 제거. 남은 5개 간선으로도 여전히 모든 꼭짓점 연결 가능.
\[
_6C_5 = 6
\]

\textbf{(iii)} \(k=4\): 간선 2개를 제거. 제거한 2개 간선이 인접하지 않으면 연결 유지. \(K_4\)에서 간선 2개를 고르는 \(_6C_2=15\)가지 모두 연결 유지.
\[
_6C_4 = 15
\]

\textbf{(iv)} \(k=3\): 간선 3개로 4개 꼭짓점을 연결. 이는 \textbf{트리}(cycle이 없는 연결 그래프) 구조여야 한다.
\begin{itemize}
\item \(K_4\)에서 3개 간선을 고르는 경우: \(_6C_3 = 20\)
\item 이 중 한 면의 3개 간선(정삼각형)을 선택하면 한 꼭짓점이 고립: 4가지
\item 연결된 경우: \(20 - 4 = 16\)
\end{itemize}

\textbf{(v)} \(k \leq 2\): 간선이 2개 이하면 4개 꼭짓점을 모두 연결 불가능 (트리 조건: \(n\)개 꼭짓점 연결에는 최소 \(n-1\)개 간선 필요).

따라서 총 연결된 경우:
\[
1 + 6 + 15 + 16 = 38
\]
\end{solution}

\begin{answer}
\hfill \boxed{38}
\end{answer}

\vspace{1cm}
\begin{center}
    \textit{Practice makes perfect!}
\end{center}

\end{document}