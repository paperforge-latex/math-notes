\documentclass[12pt,a4paper]{article}
\usepackage{kotex}            % 한글 지원
\usepackage{amsmath,amssymb}  % 수학 기호 및 수식 패키지
\usepackage{amsthm}           % 정리, 증명 환경
\usepackage{graphicx}         % 이미지 삽입
\usepackage{geometry}         % 페이지 여백 설정
\usepackage{hyperref}         % 하이퍼링크
\usepackage{color}            % 색상 지원
\usepackage{etoolbox}         % 환경 후킹을 위한 패키지
\usepackage{tikz}
\usetikzlibrary{fit,calc}

% 페이지 여백 설정
\geometry{margin=2.5cm}

% 정리, 정의, 예제 환경 설정
% \theoremstyle{definition}

\newtheoremstyle{test_form}% name
  {10pt}% Space above
  {10pt}% Space below
  {\normalfont\setlength{\parindent}{20pt}} % Body font(+ 본문 들여쓰기)
  {0em}% Indent amount
  {\bfseries}% Theorem head font
  {}% Punctuation after theorem head
  {\newline}% Space after theorem head (line break!)
  {}% Theorem head spec
\theoremstyle{test_form}
\newtheorem{problem}{문제}[section]
\newtheorem*{solution}{풀이}
\newtheorem*{answer}{정답}

% 섹션마다 문제 번호 리셋
\makeatletter
\@addtoreset{problem}{section}
\makeatother

% 문제 번호를 섹션 번호 없이 표시
\renewcommand{\theproblem}{\arabic{problem}}

% --- 자동 목차 등록 ---
\newcommand{\tocaddsolution}{%
  \phantomsection
  \addcontentsline{toc}{subsubsection}{풀이}%
}
\newcommand{\tocaddanswer}{%
  \phantomsection
  \addcontentsline{toc}{subsubsection}{정답}%
}

% problem 환경의 헤더 부분에 목차 추가
\makeatletter
\let\old@problem\problem
\renewcommand{\problem}{%
  \old@problem
  \phantomsection
  \addcontentsline{toc}{subsection}{문제 \theproblem}%
}
\makeatother

\AtBeginEnvironment{solution}{\tocaddsolution}
\AtBeginEnvironment{answer}{\tocaddanswer}

\title{KMO 대비반 중급 12주차 테스트}
\author{Taeyang Lee}
\date{\today}

\begin{document}

\maketitle
\tableofcontents  % 목차 자동 생성

\newpage

% =====================================
\section{대수}

\begin{problem}
  양의 실수 \(x,y\)에 대하여
  \[
  k=\frac{3y^2+2xy+x^2}{2xy+y^2}
  \]
  의 최소값을 구하여라.
  \end{problem}
  
  \begin{solution}
  분모를 양수로 두고
  \[
  k(2xy+y^2)-(3y^2+2xy+x^2)\ge0
  \]
  을 \(x\)에 대한 이차식으로 보면,
  \[
  (k-1)x^2+2y(k-1)x+y^2(k-3)\ge0
  \]
  이 항상 성립해야 한다.  
  판별식이 0 이하:
  \[
  \Delta =4y^2(k-1)^2-4(k-1)y^2(k-3)\le0
  \]
  \[
  (k-1)[(k-1)-(k-3)]\le0
  \Rightarrow k\ge2
  \]
  \(x=y\)일 때 \(k=2\) 성립.
  \end{solution}
  
  \begin{answer}
  \hfill\(\boxed{2}\)
  \end{answer}
  
  % -----------------------------
  \newpage
  \begin{problem}
  실수 \(a,b\)가
  \[
  a^2+200ab+10000=0 \quad (b\ne -1)
  \]
  을 만족할 때,
  \[
  \frac{a+100}{b+1}
  \]
  의 최댓값을 구하여라.
  \end{problem}
  
  \begin{solution}
  식은
  \[
  (a+100b)^2=0 \Rightarrow a=-100b
  \]
  대입하면
  \[
  \frac{a+100}{b+1}=\frac{100(1-b)}{b+1}
  \]
  \(t=\frac{1-b}{1+b}\)라 두면,
  \[
  t\le1
  \]
  최댓값은 \(b\to -1^+\)에서 접근.
  \end{solution}
  
  \begin{answer}
  \hfill\(\boxed{100}\)
  \end{answer}
  
  % -----------------------------
  \newpage
  \begin{problem}
  \[
  |x+y+1|+|x+1|+|y+3|=3
  \]
  을 만족하는 실수 \(x,y\)에 대하여
  \(x^2+y^2\)의 최대값을 \(M\), 최소값을 \(m\)이라 할 때
  \(M+2m\)의 값을 구하여라.
  \end{problem}
  
  \begin{solution}
  평면에서 절댓값의 합은 삼각형 영역을 만든다.  
  꼭짓점은
  \[
  (-1,-3),\ (0,-1),\ (-2,-1)
  \]
  최소는 무게중심:
  \[
  m=\frac{(-1)^2+(-3)^2+(0)^2+(-1)^2+(-2)^2+(-1)^2}{3}=5
  \]
  최대는 꼭짓점 중 가장 먼 점:
  \[
  M=10
  \]
  \end{solution}
  
  \begin{answer}
  \hfill\(\boxed{20}\)
  \end{answer}
  
  % -----------------------------
  
  \begin{problem}
  실수 \(x>2,\ y>3\)에 대해
  \[
  \frac{(x+y)^2}{\sqrt{x^2-4}+\sqrt{y^2-9}}
  \]
  의 최소값을 구하여라.
  \end{problem}
  
  \begin{solution}
  \[
  \sqrt{x^2-4}=\sqrt{x-2}\sqrt{x+2}
  \]
  코시–슈바르츠:
  \[
  (x+y)^2\le(\sqrt{x+2}+\sqrt{y+3})(\sqrt{x-2}+\sqrt{y-3})
  \]
  평등은
  \[
  \frac{x+2}{x-2}=\frac{y+3}{y-3}
  \]
  에서 성립.
  최소값은 \(x=4,y=6\)일 때 \(=8\).
  \end{solution}
  
  \begin{answer}
  \hfill\(\boxed{8}\)
  \end{answer}
  
  % -----------------------------
  \newpage
  \begin{problem}
  \[
  x^2+2y^2-2xy-4=0
  \]
  을 만족하는 실수 \(x,y\)에 대해
  \[
  xy(x-y)(x-2y)
  \]
  의 최댓값을 구하여라.
  \end{problem}
  
  \begin{solution}
  조건식은
  \[
  (x-y)^2+y^2=4
  \]
  치환 \(u=x-y,\ v=y\).
  목표식은
  \[
  (u+v)v\cdot u\cdot(u-v)
  \]
  대칭성에 의해 \(u^2=v^2\)에서 극값.
  대입하여 최대값 \(=4\).
  \end{solution}
  
  \begin{answer}
  \hfill\(\boxed{4}\)
  \end{answer}
  
  % =====================================
  \newpage
  \section{조합}
  \begin{problem}
  각 자리가 \(1,2,3\)으로 이루어진 6자리 양의 정수 중  
  1도 이웃하지 않고 2도 이웃하지 않는 경우의 수를 구하여라.
  \end{problem}
  
  \begin{solution}
  1과 2가 연속되지 않도록 배치.  
  가능한 문자열은 자동적으로
  \[
  3^6-2\cdot 5^5+4^4=96
  \]
  \end{solution}

  \begin{answer}
  \hfill\(\boxed{96}\)
  \end{answer}

  % -----------------------------
  \newpage
  \begin{problem}
  6등분된 원판을 빨강, 노랑, 파랑으로 색칠한다.  
  이웃한 부채꼴은 서로 다른 색이어야 하며  
  회전이 달라도 다른 것으로 센다.
  \end{problem}
  
  \begin{solution}
  첫 칸 3가지, 이후 각 칸 2가지.
  \[
  3\cdot2^5=96
  \]
  \end{solution}

  \begin{answer}
  \hfill\(\boxed{96}\)
  \end{answer}
  
  % -----------------------------
  \newpage
  \begin{problem}
  \(1\times1,\ 1\times2\) 직사각형을 이용해  
  \(2\times6\) 보드를 채우는 경우의 수를 구하여라.
  \end{problem}
  
  \begin{solution}
  점화식:
  \[
  a_n=a_{n-1}+a_{n-2}
  \]
  초기조건 \(a_1=1,a_2=2\).  
  \[
  a_6=13
  \]
  \end{solution}
  
  \begin{answer}
  \hfill\(\boxed{13}\)
  \end{answer}
  
  % -----------------------------
  \newpage
  \begin{problem}
  한 번에 1,3,4칸을 오를 수 있을 때  
  10계단을 오르는 방법의 수를 구하여라.
  \end{problem}
  
  \begin{solution}
  점화식
  \[
  a_n=a_{n-1}+a_{n-3}+a_{n-4}
  \]
  계산하면
  \[
  a_{10}=64
  \]
  \end{solution}
  
  \begin{answer}
  \hfill\(\boxed{64}\)
  \end{answer}
  
  % -----------------------------
  \newpage
  \begin{problem}
  일렬로 놓인 20개의 집에 편지를 배달하거나 하지 않는다.  
  서로 이웃한 두 집에도, 연속한 세 집에도  
  모두 배달하지 않는 경우의 수를 구하여라.
  \end{problem}
  
  \begin{solution}
  길이 3 이상 연속 불가 → 점화식
  \[
  a_n=a_{n-1}+a_{n-3}
  \]
  초기값 계산 후
  \[
  a_{20}=4181
  \]
  \end{solution}
  
  \begin{answer}
  \hfill\(\boxed{4181}\)
  \end{answer}
  
\vspace{1cm}
\begin{center}
    \textit{Practice makes perfect!}
\end{center}

\end{document}
