\documentclass[12pt,a4paper]{article}
\usepackage{kotex}            % 한글 지원
\usepackage{amsmath,amssymb}  % 수학 기호 및 수식 패키지
\usepackage{amsthm}           % 정리, 증명 환경
\usepackage{graphicx}         % 이미지 삽입
\usepackage{geometry}         % 페이지 여백 설정
\usepackage{hyperref}         % 하이퍼링크
\usepackage{color}            % 색상 지원
\usepackage{etoolbox}         % 환경 후킹을 위한 패키지
\usepackage{tikz}
\usetikzlibrary{fit,calc}

% 페이지 여백 설정
\geometry{margin=2.5cm}

% 정리, 정의, 예제 환경 설정
% \theoremstyle{definition}

\newtheoremstyle{test_form}% name
  {10pt}% Space above
  {10pt}% Space below
  {\normalfont\setlength{\parindent}{20pt}} % Body font(+ 본문 들여쓰기)
  {0em}% Indent amount
  {\bfseries}% Theorem head font
  {}% Punctuation after theorem head
  {\newline}% Space after theorem head (line break!)
  {}% Theorem head spec
\theoremstyle{test_form}
\newtheorem{problem}{문제}[section]
\newtheorem*{solution}{풀이}
\newtheorem*{answer}{정답}

% 섹션마다 문제 번호 리셋
\makeatletter
\@addtoreset{problem}{section}
\makeatother

% 문제 번호를 섹션 번호 없이 표시
\renewcommand{\theproblem}{\arabic{problem}}

% --- 자동 목차 등록 ---
\newcommand{\tocaddsolution}{%
  \phantomsection
  \addcontentsline{toc}{subsubsection}{풀이}%
}
\newcommand{\tocaddanswer}{%
  \phantomsection
  \addcontentsline{toc}{subsubsection}{정답}%
}

% problem 환경의 헤더 부분에 목차 추가
\makeatletter
\let\old@problem\problem
\renewcommand{\problem}{%
  \old@problem
  \phantomsection
  \addcontentsline{toc}{subsection}{문제 \theproblem}%
}
\makeatother

\AtBeginEnvironment{solution}{\tocaddsolution}
\AtBeginEnvironment{answer}{\tocaddanswer}

\title{}
\author{Taeyang Lee}
\date{\today}

\begin{document}

\maketitle
\tableofcontents  % 목차 자동 생성

\newpage

% =====================================
\section{대수}

\begin{problem}
\setlength{\parindent}{0pt}

어떤 부자가 자신의 세 아들에게 다음과 같은 유언을 남겼다.
본인이 가지고 있는 낙타 17마리에 대해
첫번째 자식에게는 본인이 가진 낙타의 $\frac{1}{2}$을
두번째 자식에게는 본인이 가진 낙타의 $\frac{1}{3}$을
세번째 자식에게는 본인이 가진 낙타의 $\frac{1}{9}$을 남기겠다.

하지만, 막상 낙타를 나누려고 하다보니, 2, 3, 9가 모두 17로는 나누어 떨어지지 않음을 알게 되었다.

이에 현자에게 어찌할지를 물으니, 본인이 가진 낙타를 한마리 추가하여 각각의 낙타를 나누고 나면 9마리, 6마리, 2마리를 나눠가지고 남게되는 한마리의 낙타는 현자가 다시 가져가면 됨을 알게되었다.

도대체 이 문제에서 어떤 일이 일어난 것일까??

\begin{flushright}(20점)\end{flushright}

\end{problem}

\begin{solution}
\setlength{\parindent}{0pt}

이 문제를 처음 보았을때, 이해되지 않는 부분은 우리가 분수식을 바탕으로 연산을 진행할 떄, 나누어 떨어지지 않는 수들이 갑자기 마술을 부린것과 같이 나누어 떨어지게 되고, 그 조작이 행위의 전후로 바뀌지 않아서라고 생각할 것이다.
하지만, 곰곰히 생각해보면 이 문제의 트릭은 
	
 \[
     \frac{1}{2} + \frac{1}{3}  + \frac{1}{9} \ne 1
 \]
이라는 것에 있다.

즉, 전체의 양이 1이라고 생각하고 하면, 이 문제가 처음부터 이상하다고 느끼겠지만,
 \[
     \frac{1}{2} + \frac{1}{3}  + \frac{1}{9} = \frac{17}{18}
 \]

임을 이용하면,
\[
	\frac {\frac{1}{2}} {\frac{17}{18}} = \frac{9}{17}
\]
가 되므로, 처음부터 9마리가 첫번째 자식에게 배정되어 있었음을 알 수 있다.

이처럼, 분수가 의미하는 것이 "비율" 임에도 불구하고 이를 단순하게 생각하지 못하여 더 확장된 사고를 하지 못하는 경우가 존재한다.


이 다음 문항에서 이 내용을 더 생각해보자

\end{solution}

\begin{answer}
\hfill \boxed{}
\end{answer}


\newpage

\begin{problem}

 \[
\frac{1}{9} = \frac{1}{10} + \frac{1}{10^2} + \frac{1}{10^3} + \cdots  
 \]

위 계산의 내용은 등비급수의 합 공식으로 익히 알고있거나,

\[
	\frac{1}{9} = 0. \dot{1}
\]

임을 이용하면 자명함을 알 수 있다.
하지만 이 상황에서도 비슷한 생각을해보자.


\begin{flushright}(20점)\end{flushright}

\end{problem}

\begin{solution}
\setlength{\parindent}{0pt}

주어진 값 1을 9명에게 나누어 줘야하는 상황을 상상하자.
그렇다면 한사람이 가지게 되는 값은 정확히 $\frac{1}{9}$가 될 것이다.
하지만 마치 현자에게 낙타를 한마리 빌려오는 것 처럼, 가상의 사람이 존재한다고 생각하자.
이 경우에 한사람이 가지게 되는 값은 기존의 $\frac{1}{9}$이 아니라 $\frac{1}{10}$이 될 것이다.
하지만 가상의 사람은 존재하지 않고, 실제로는 $\frac{1}{10}$이 남은 상황으로 생각할 수 있다.
즉 다시말해 남아 있는 $\frac{1}{10}$을 다시 9명에게 나눠주는 문제로 상황이 바뀌었다.
그렇다면, $\frac{1}{10}$을 또 9명에게 나누어주면 되는데,

 \[
\frac{1}{9} = \frac{1}{10} + \frac{1}{10 \times 9}
 \]

이때에도, 가상의 사람을 상상하여, 10명에게 나눠준다고 생각하면 이 과정을 무한히 반복할 수 있을 것이다.

이러한 생각으로 무한등비급수의 내용을 간략한 분수식의 연산으로 생각해볼 수 있다.

\end{solution}

\begin{answer}
\hfill \boxed{}
\end{answer}




\vspace{1cm}
\begin{center}
  \textit{Practice makes perfect!}
\end{center}


\end{document}

